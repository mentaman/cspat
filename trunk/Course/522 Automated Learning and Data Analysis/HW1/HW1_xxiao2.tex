%% Based on a TeXnicCenter-Template by Gyorgy SZEIDL.
%%%%%%%%%%%%%%%%%%%%%%%%%%%%%%%%%%%%%%%%%%%%%%%%%%%%%%%%%%%%%

%------------------------------------------------------------
%
\documentclass{article}%
%Options -- Point size:  10pt (default), 11pt, 12pt
%        -- Paper size:  letterpaper (default), a4paper, a5paper, b5paper
%                        legalpaper, executivepaper
%        -- Orientation  (portrait is the default)
%                        landscape
%        -- Print size:  oneside (default), twoside
%        -- Quality      final(default), draft
%        -- Title page   notitlepage, titlepage(default)
%        -- Columns      onecolumn(default), twocolumn
%        -- Equation numbering (equation numbers on the right is the default)
%                        leqno
%        -- Displayed equations (centered is the default)
%                        fleqn (equations start at the same distance from the right side)
%        -- Open bibliography style (closed is the default)
%                        openbib
% For instance the command
%           \documentclass[a4paper,12pt,leqno]{article}
% ensures that the paper size is a4, the fonts are typeset at the size 12p
% and the equation numbers are on the left side
%
\usepackage{amsmath}%
\usepackage{amsfonts}%
\usepackage{amssymb}%
\usepackage{graphicx}
%-------------------------------------------
\newtheorem{theorem}{Theorem}
\newtheorem{acknowledgement}[theorem]{Acknowledgement}
\newtheorem{algorithm}[theorem]{Algorithm}
\newtheorem{axiom}[theorem]{Axiom}
\newtheorem{case}[theorem]{Case}
\newtheorem{claim}[theorem]{Claim}
\newtheorem{conclusion}[theorem]{Conclusion}
\newtheorem{condition}[theorem]{Condition}
\newtheorem{conjecture}[theorem]{Conjecture}
\newtheorem{corollary}[theorem]{Corollary}
\newtheorem{criterion}[theorem]{Criterion}
\newtheorem{definition}[theorem]{Definition}
\newtheorem{example}[theorem]{Example}
\newtheorem{exercise}[theorem]{Exercise}
\newtheorem{lemma}[theorem]{Lemma}
\newtheorem{notation}[theorem]{Notation}
\newtheorem{problem}[theorem]{Problem}
\newtheorem{proposition}[theorem]{Proposition}
\newtheorem{remark}[theorem]{Remark}
\newtheorem{solution}[theorem]{Solution}
\newtheorem{summary}[theorem]{Summary}
\newenvironment{proof}[1][Proof]{\textbf{#1.} }{\ \rule{0.5em}{0.5em}}

\begin{document}

\begin{flushleft}
\textbf{Course:} CSC522, Automated Learning and Data Analysis\\
\textbf{Homework 1}\\
\textbf{Student: Xusheng Xiao}\\
\textbf{Unity ID: xxiao2} \\
\textbf{Email: xxiao2@ncsu.edu}
\end{flushleft}

\noindent{\hrulefill}

\bigskip

\begin{enumerate}
	\item \textbf{Problem 1 (6 points).} Please classify the following attributes as binary, discrete, or continuous. Also classify them as qualitative (nominal or ordinal) or quantitative (interval or ratio). Some cases may have more than one interpretation, so briefly indicate your reasoning then in doubt.	
	\begin{enumerate}
		\item Time in terms of AM or PM \\
		Binary and nomial. 	
		
		\item Brightness as measured by a light meter \\
        Continuous and ratio.		
		
		\item Brightness as measured by people’s judgments \\
	    Discrete and ordinal. People may specify it like "dark, bright, very bright". Since it still has an way to order the brightness, it should be "discrete and ordinal"	
		
		\item Angles as measured in degrees between 0 and 360. \\
		Continuous and radio.
		
		\item Bronze, Silver, and Gold medals as awarded at the Olympics \\
		Discrete and nominal.		
		
		\item Height above sea level. \\
		Continuous and ratio.
		
		\item Number of patients in a hospital \\
		Discrete and radio.		
		
		\item ISBN numbers for books. (Look up the format on the Web.) \\
		Discrete and nominal.		
		
		\item Ability to pass light in terms of the following values: opaque, translucent, transparent. \\
		Discrete and ordinal.
		
		\item Military rank\\
		Discrete and ordinal.		
		
		\item Distance from the center of campus. \\
		Continuous and ratio.	
		
		\item Density of a substance in grams per cubic centimeter. \\
		Continuous and ratio.
		
		\item Coat check number. (When you attend an event and you give your coat in exchange to a number that you can use to claim your coat when you leave.) \\
		Discrete and nominal.
	\end{enumerate}
  
 \item \textbf{Problem 2 (5 points).} For the following vectors, x and y, calculate the indicated similarity or distance measures.
 \begin{enumerate}
		\item X=(1,1,1,1), Y=(2,2,2,2): cosine, correlation coefficient, Euclidean distance \\
		\textbf{cosine distance:} \\
		$ x \cdot  y = 1*2 + 1*2 + 1*2 + 1*2  = 8$ \\
		$ ||x|| = \sqrt{1*1 + 1*1 + 1*1 + 1*1} = 2 $ \\
		$ ||y|| = \sqrt{2*2 + 2*2 + 2*2 + 2*2} = 4 $ \\
		$ cos(x,y) = \dfrac{x \cdot y }{ ||x|| ||y||} = \dfrac{8}{2*4} = 1 $\\
		
		\textbf{correlation coefficient: }\\
		$ \bar{x} = \dfrac{1}{4} \sum_{k=1}^4x_k=1, \bar{y} = \dfrac{1}{4} \sum_{k=1}^4y_k = 2 $ \\
		$ covariance(x,y) = s_{xy} = \dfrac{1}{4 - 1} \sum_{k=1}^{4}(x_k - \bar{x})(y_k-\bar{y}) = 0$	\\
		$ standard\_deviation(x) = s_{x} = \sqrt{\dfrac{1}{4-1}\sum_{k=1}^{4}(x_k - \bar{x})^2} = 0$	\\		
		$ standard\_deviation(y) = s_{y} = \sqrt{\dfrac{1}{4-1}\sum_{k=1}^{4}(y_k - \bar{y})^2} = 0$	\\		
		$ corr(x,y) = \dfrac{covariance(x,y)}{standard\_deviation(x)*standard\_deviation(y)} = \dfrac{s_{xy}}{s_x*s_y} = \dfrac{0}{0}$	 \\
		
		\textbf{Euclidean distance} = $\sqrt{(1-2)^2 + (1-2)^2 + (1-2)^2 + (1-2)^2 }$ = 2\\
			
		\item X=(0,1,0,1), Y=(1,0,1,0): cosine, correlation coefficient, Euclidean distance, Jaccard coefficient \\
		\textbf{cosine distance:} \\
		$ x \cdot  y = 0*1+1*0+0*1+1*0 = 0$ \\
		$ ||x|| = \sqrt{0*0 + 1*1 + 0*0 + 1*1} = \sqrt{2} $ \\
		$ ||y|| = \sqrt{1*1 + 0*0 + 1*1 + 0*0} = \sqrt{2} $ \\
		$ cos(x,y) = \dfrac{x \cdot y }{ ||x|| ||y||} = \dfrac{0}{\sqrt{2}*\sqrt{2}} = 0 $\\
		
		
		\textbf{correlation coefficient:}\\
		$ \bar{x} = \dfrac{1}{4} \sum_{k=1}^4x_k=0.5, \bar{y} = \dfrac{1}{4} \sum_{k=1}^4y_k = 0.5 $ \\
		$ covariance(x,y) = s_{xy} = \dfrac{1}{4 - 1} \sum_{k=1}^{4}(x_k - \bar{x})(y_k-\bar{y}) = -\dfrac{1}{3}$	\\
		$ standard\_deviation(x) = s_{x} = \sqrt{\dfrac{1}{4-1}\sum_{k=1}^{4}(x_k - \bar{x})^2} = \sqrt{\dfrac{1}{3}}$	\\		
		$ standard\_deviation(y) = s_{y} = \sqrt{\dfrac{1}{4-1}\sum_{k=1}^{4}(y_k - \bar{y})^2} = \sqrt{\dfrac{1}{3}}$	\\		
		$ corr(x,y) = \dfrac{covariance(x,y)}{standard\_deviation(x)*standard\_deviation(y)} = \dfrac{s_{xy}}{s_x*s_y} = -1$	 \\
		
				
		\textbf{Euclidean distance} = $\sqrt{(0-1)^2 + (1-0)^2 + (0-1)^2 + (1-0)^2 }$ = 2\\	

		\textbf{Jaccard coefficient} = $\dfrac{f_{11}}{f_{01}+f_{10}+f_{11}} = \dfrac{0}{2+2+0}$ = 0\\	
		
		\item X=(0,-1,0,1), Y=(1,0,-1,0): cosine, correlation, Euclidean distance \\
		\textbf{cosine distance:} \\
		$ x \cdot  y = 0*1 + -1*0 + 0*-1 + 1*0  = 0$ \\
		$ ||x|| = \sqrt{0*0 + -1*-1 + 0*0 + 1*1} = \sqrt{2} $ \\
		$ ||y|| = \sqrt{1*1 + 0*0 + -1*-1 + 0*0} = \sqrt{2} $ \\
		$ cos(x,y) = \dfrac{x \cdot y }{ ||x|| ||y||} = \dfrac{0}{2} = 0 $\\
		
		\textbf{correlation coefficient: }\\
		$ \bar{x} = \dfrac{1}{4} \sum_{k=1}^4x_k=0, \bar{y} = \dfrac{1}{4} \sum_{k=1}^4y_k = 0 $ \\
		$ covariance(x,y) = s_{xy} = \dfrac{1}{4 - 1} \sum_{k=1}^{4}(x_k - \bar{x})(y_k-\bar{y}) = 0$	\\
		$ standard\_deviation(x) = s_{x} = \sqrt{\dfrac{1}{4-1}\sum_{k=1}^{4}(x_k - \bar{x})^2} = \sqrt{\dfrac{2}{3}}$	\\		
		$ standard\_deviation(y) = s_{y} = \sqrt{\dfrac{1}{4-1}\sum_{k=1}^{4}(y_k - \bar{y})^2} = \sqrt{\dfrac{2}{3}}$	\\		
		$ corr(x,y) = \dfrac{covariance(x,y)}{standard\_deviation(x)*standard\_deviation(y)} = \dfrac{s_{xy}}{s_x*s_y} = \dfrac{0}{\dfrac{2}{3}}$	 \\
		
		\textbf{Euclidean distance} = $\sqrt{(0-1)^2 + (-1-0)^2 + (0-(-1))^2 + (1-0)^2 }$ = 2\\
	
		
		\item X=(1,1,0,1,0,1), Y=(1,1,1,0,0,1): cosine, correlation coefficient, Jaccard coefficient \\
		\textbf{cosine distance:} \\
		$ x \cdot  y = 1*1+1*1+0*1+1*0 + 0*0+1*1 = 3$ \\
		$ ||x|| = \sqrt{1*1+1*1+0*0 + 1*1 + 0*0 + 1*1} = 2 $ \\
		$ ||y|| = \sqrt{1*1 +1*1 +1*1 + 0*0 + 0*0 + 1*1 } = 2 $ \\
		$ cos(x,y) = \dfrac{x \cdot y }{ ||x|| ||y||} = \dfrac{3}{2*2} =\dfrac{3}{4} $\\
		
		
		\textbf{correlation coefficient:}\\
		$ \bar{x} = \dfrac{1}{6} \sum_{k=1}^4x_k=\dfrac{2}{3}, \bar{y} = \dfrac{1}{6} \sum_{k=1}^4y_k = \dfrac{2}{3} $ \\
		$ covariance(x,y) = s_{xy} = \dfrac{1}{6 - 1} \sum_{k=1}^{6}(x_k - \bar{x})(y_k-\bar{y}) = \dfrac{1}{15}$	\\
		$ standard\_deviation(x) = s_{x} = \sqrt{\dfrac{1}{6-1}\sum_{k=1}^{6}(x_k - \bar{x})^2} = \sqrt{\dfrac{4}{15}}$	\\		
		$ standard\_deviation(y) = s_{y} = \sqrt{\dfrac{1}{6-1}\sum_{k=1}^{6}(y_k - \bar{y})^2} = \sqrt{\dfrac{4}{15}}$	\\		
		$ corr(x,y) = \dfrac{covariance(x,y)}{standard\_deviation(x)*standard\_deviation(y)} = \dfrac{s_{xy}}{s_x*s_y} = \dfrac{1}{4}$	 \\
		
				
		\textbf{Jaccard coefficient} = $\dfrac{f_{11}}{f_{01}+f_{10}+f_{11}} = \dfrac{3}{1+1+3} = \dfrac{3}{5}$\\	
		
		\item X=(2,-1,0,2,0,-3), Y=(-1,1,-1,0,0,-1): cosine, correlation coefficient \\
		\textbf{cosine distance:} \\
		$ x \cdot  y = 2*-1+-1*1+0*-1+2*0 + 0*0+-3*-1 = 0$ \\
		$ ||x|| = \sqrt{2*2+-1*-1+0*0 + 2*2 + 0*0 + -3*-3} = \sqrt{18} $ \\
		$ ||y|| = \sqrt{-1*-1 +1*1 +-1*-1 + 0*0 + 0*0 + -1*-1 } = 2 $ \\
		$ cos(x,y) = \dfrac{x \cdot y }{ ||x|| ||y||} = \dfrac{0}{18*2} =0$\\
		
		
		\textbf{correlation coefficient:}\\
		$ \bar{x} = \dfrac{1}{6} \sum_{k=1}^4x_k=0, \bar{y} = \dfrac{1}{6} \sum_{k=1}^4y_k = -\dfrac{1}{3} $ \\
		$ covariance(x,y) = s_{xy} = \dfrac{1}{6 - 1} \sum_{k=1}^{6}(x_k - \bar{x})(y_k-\bar{y}) = 0$	\\
		$ standard\_deviation(x) = s_{x} = \sqrt{\dfrac{1}{6-1}\sum_{k=1}^{6}(x_k - \bar{x})^2} = \sqrt{\dfrac{18}{5}}$	\\		
		$ standard\_deviation(y) = s_{y} = \sqrt{\dfrac{1}{6-1}\sum_{k=1}^{6}(y_k - \bar{y})^2} = \sqrt{\dfrac{2}{3}}$	\\		
		$ corr(x,y) = \dfrac{covariance(x,y)}{standard\_deviation(x)*standard\_deviation(y)} = \dfrac{s_{xy}}{s_x*s_y} = 0$	 \\	
	
		
 \end{enumerate}
 
 \item\textbf{Problem 3 (6 points). } Show that the set difference metric given by $D(A,B) := size(A-B) + size(B-A)$ satisfies the metric axioms given on pages 70/71 of our textbook. Here, $A$ and $B$ are sets, and $A-B$ indicates the set difference. \\
 
 
 \begin{theorem}
    $D(A,B) := size(A-B) + size(B-A)$ satisfies the metric axioms, where A and B are sets, and A-B indicates the set difference.
 \end{theorem}
 \begin{proof}
 To prove $D(A,B) := size(A-B) + size(B-A)$ satisfies the metric axioms, we need to show three properties of metrics hold for $D(A,B)$. \\
 \begin{enumerate}
\item \textbf{Positivity.} 
\begin{enumerate}
\item $D(A,A) \geq 0 $ for all $A$. Since $ A - B $ indicates the set difference, $A - A = \emptyset \Rightarrow size(A - A) = 0$.\\
 As a result, $D(A,A) := size(A-A) + size(A-A) = 0 + 0 = 0 \Rightarrow D(A,A) \geq 0$.
\item $D(A,B) = 0 $ only if $ A = B $. Suppose $A \neq B$, then $A-B \neq \emptyset \Rightarrow size(A-B) > 0, size(B-A) >0.$ \\
As a result, $D(A,B) := size(A-B) + size(B-A) > 0 \Rightarrow D(A,B) \neq 0$. \\
Since $D(A,A) \geq 0 $ for all $A$, $D(A,B) = 0 $ only if $ A = B $.
\end{enumerate}

\item \textbf{Symmetry.} \\
$ D(A,B) = D(B,A)  $ for all $ x $ and $ y $. Since $D(A,B) = size(A-B) + size(B-A)$, $D(B,A) = size(B-A) + size(A-B) = D(A,B)$.

\item \textbf{Triangle Inequality.} \\
$ D(A,C) \leq D(A,B) + D(B,C) $ for all set $A,B,$ and $C$. \\
By De Morgan, $D(A,B) = size(A-B) + size(B-A) = size(A) + size(B) - 2size(A \bigcap B) $ \\
 $D(B,C) = size(B-C) + size(C-B) = size(B) + size(C) - 2size(B \bigcap C) $ \\
 $D(A,C) = size(A-C) + size(C-A) = size(A) + size(C) - 2size(A \bigcap C) $.\\
 $D(A,B) + D(B,C) - D(A,C)=size(A) + size(B) - 2size(A \bigcap B) + size(B) + size(C) - 2size(B \bigcap C)  - size(A) - size(C) + 2size(A \bigcap C)  $ \\
 $\Rightarrow D(A,B) + D(B,C) - D(A,C)= 2size(B) - 2size(A \bigcap B) - 2size(B \bigcap C) + 2size(A \bigcap C)  $ \\
 By De Morgan, $size(B) + size(A \bigcap B \bigcap C) \geq size(A \bigcap B) + size(B \bigcap C) $. \\
 Since $size(A \bigcap C)  \geq  size(A \bigcap B \bigcap C) $, $size(B) + size(A \bigcap C) \geq size(A \bigcap B) + size(B \bigcap C) $.\\
 $\Rightarrow 2size(B) - 2size(A \bigcap B) - 2size(B \bigcap C) + 2size(A \bigcap C)  \geq 0$ \\
 $\Rightarrow D(A,B) + D(B,C) - D(A,C) \geq 0 \Rightarrow D(A,C) \leq D(A,B) + D(B,C) $\\
 Thus, $ D(A,C) \leq D(A,B) + D(B,C) $ for all set $A,B,$ and $C$.
\end{enumerate}
 Since all these three properties hold for $D(A,B) := size(A-B) + size(B-A)$, $D(A,B)$ satisfies the metric axioms.
 
 \end{proof}
 


\item \textbf{Problem 4 (6 points).} Describe how you would create visualizations to display information that describes
 \begin{enumerate}
		\item Computer networks. Be sure to include both the static aspects of the network, such as connectivity, and the dynamic aspects, such as traffic. 		
		\item The distribution of specific plant and animal species around the world for a specific moment in time.
		\item The use of computer resources such as processor time, main memory, and disk for asset of benchmark database programs.
 \end{enumerate}
 
In your answers, please address the following issues:
\begin{itemize}
\item Representation: how will you map objects, attributes, and relationships to visual elements?
\item Arrangement: are there special considerations that need to be taken into account with respect to how visual elements are displayed? E.g. choice of viewpoint, use of transparency, etc.
\item Selection: how will you handle a large number of attributes and data objects?
\end{itemize}

\begin{enumerate}
\item \textbf{Computer network.} Computer networks can be visualized using the graph representation. Each computer, server, router or gateway will be the nodes. If two computers are directly connected, there will be a link between the nodes. The width of the link can represent the bandwidth and the color can represent the busy level of the connection.
\item \textbf{Distribution of plant and animal species.} The species can be represented using tree. However, showing all the species in a whole tree does not make sense, since the number is too huge. Thus, we can only show the first several high level names of the species and users can expand the specific specie if they are interested. Besides, we can also using color for representing the number of animal species in one tree node.

\item \textbf{Computer Resources.} The computer resources can be represented using bar plot. To combine the processor time, memory and disk, we can use three bars in different colors for a database program.

\end{enumerate}

		

\item \textbf{Problem 5 (5 points).}
 \begin{enumerate}
		\item Describe how a box plot can give information about whether the value of an attribute is symmetrically distributed. What can you say about the symmetry of the distribution of the attributes shown in Figure 3.11 on page 115 of our textbook. \\

		In the box plot, the lower and upper ends of the box indicate the 25th and 75th percentiles, which are the median of the lower and upper halves.  The line inside the box indicate the 50th percentiles. If the line is in the middle of the box, which means that the lower half and the upper half have similar numbers, the values of the attribute is symmetrically distributed. Similarly, if the line is near the lower end or upper end, the distribution is not symmetric.

		\item Compare sepal length, sepal width, petal length, and petal width using Figure 3.12 on page 115. \\
		
		Setosa: sepal length $>$ sepal width $>$ petal length $>$ petal width. \\
		Versicolour and Virginiica: sepal length $>$ sepal width, petal length $>$ petal width, sepal length $>$ petal length, sepal width $>$ petal width.
 \end{enumerate}



\end{enumerate}
\end{document}
