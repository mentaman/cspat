\section{Introduction} 
Software Engineering is a knowledge-intensive activity, presumably requiring intelligence. Many software engineering activities, such as testing, analysis and debugging, require intensive human intelligence and are error-prone. To reduce human efforts in the activities of software engineering, Artificial Intelligence (AI) techniques, which aims to create computer systems that exhibit some form of human intelligence, are employed to assist or automate various activities of software engineering, such as testing, program analysis, debugging and even self-repair. In this report, we present the details of the application of AI in software engineering and provide a detailed survey of the existing tools and techniques associating with AI in three important software engineering activities: testing, fault detection and software repair.

Over the past decades, many AI techniques are applied to assist automated software testing, such as constraint solving~\cite{constraintsolving} used in Dynamic Symbolic Execution~\cite{symbolic, dart, cute} for test-input generation, heuristics used to prune search space of test-generation tools~\cite{prune,fitness}, and machine learning used in statistical software testing~\cite{mlinstatistics} and coverage prediction of testing tools~\cite{predictCoverage}. In this report, we provide the details on test generation using symbolic execution and pruning search space of symbolic execution using heuristics.

As the size of complexity of software has grown quickly in past decades, the difficulty of finding and fixng bugs has increased. Recent research works~\cite{wrongDefinition,online} that use AI techniques have advanced the research in reducing the human efforts on fault detection: Shi et al.~\cite{wrongDefinition} proposes an approach to first learn the Definition-Use Invariants and then use the learned knowledge of Definition-Use for detecting concurrency and sequential bugs; Baah et al~\cite{online} and proposes a new machine-learning technique that performs fault detection for deployed software. I plan to study in details how these research works adopt the concepts and techniques of AI to assist the task of fault detection.

By adapting the well-known AI techniques, even the most challenging tasks, debugging and self-repair, can be half or even full automated. Genetic programming, an evolutionary algorithm-based methodology in AI, is used and adapted by Weimer et al.~\cite{geneticPatch}' approach to automatically find patches for programs and automatically fix bugs~\cite{repair}. Inspired by these works, Schulte et al. further propose an approach to study the evolution of assembly code~\cite{evolutionaryComputation} for automated program repair. I plan to investigate these techniques to study how AI can achieve automated software repair.



