%% Based on a TeXnicCenter-Template by Gyorgy SZEIDL.
%%%%%%%%%%%%%%%%%%%%%%%%%%%%%%%%%%%%%%%%%%%%%%%%%%%%%%%%%%%%%

%------------------------------------------------------------
%
\documentclass{article}%
%Options -- Point size:  10pt (default), 11pt, 12pt
%        -- Paper size:  letterpaper (default), a4paper, a5paper, b5paper
%                        legalpaper, executivepaper
%        -- Orientation  (portrait is the default)
%                        landscape
%        -- Print size:  oneside (default), twoside
%        -- Quality      final(default), draft
%        -- Title page   notitlepage, titlepage(default)
%        -- Columns      onecolumn(default), twocolumn
%        -- Equation numbering (equation numbers on the right is the default)
%                        leqno
%        -- Displayed equations (centered is the default)
%                        fleqn (equations start at the same distance from the right side)
%        -- Open bibliography style (closed is the default)
%                        openbib
% For instance the command
%           \documentclass[a4paper,12pt,leqno]{article}
% ensures that the paper size is a4, the fonts are typeset at the size 12p
% and the equation numbers are on the left side
%
\usepackage{amsmath}%
\usepackage{amsfonts}%
\usepackage{amssymb}%
\usepackage{graphicx}
%-------------------------------------------
\newtheorem{theorem}{Theorem}
\newtheorem{acknowledgement}[theorem]{Acknowledgement}
\newtheorem{algorithm}[theorem]{Algorithm}
\newtheorem{axiom}[theorem]{Axiom}
\newtheorem{case}[theorem]{Case}
\newtheorem{claim}[theorem]{Claim}
\newtheorem{conclusion}[theorem]{Conclusion}
\newtheorem{condition}[theorem]{Condition}
\newtheorem{conjecture}[theorem]{Conjecture}
\newtheorem{corollary}[theorem]{Corollary}
\newtheorem{criterion}[theorem]{Criterion}
\newtheorem{definition}[theorem]{Definition}
\newtheorem{example}[theorem]{Example}
\newtheorem{exercise}[theorem]{Exercise}
\newtheorem{lemma}[theorem]{Lemma}
\newtheorem{notation}[theorem]{Notation}
\newtheorem{problem}[theorem]{Problem}
\newtheorem{proposition}[theorem]{Proposition}
\newtheorem{remark}[theorem]{Remark}
\newtheorem{solution}[theorem]{Solution}
\newtheorem{summary}[theorem]{Summary}
\newenvironment{proof}[1][Proof]{\textbf{#1.} }{\ \rule{0.5em}{0.5em}}

\begin{document}

\begin{flushleft}
\textbf{Course:} CSC707, Automata, Computability and Computational Theory\\
\textbf{Homework 6}: Context-free languages, context-free grammars, PDA, Pumping lemma. \\
\textbf{Submission:} Use Wolfware\\
\textbf{File Format:} LaTeX and PDF, and any images you have\\
\end{flushleft}

\begin{center}
\fbox{\textbf{Due Date:} \textbf{2:00 A.M. (EST), Tuesday, April 13, 2010}}\\
\end{center}

\noindent{\hrulefill}

\bigskip

\begin{enumerate}

	\item Prove non-context-free using Pumping lemma:
	\begin{enumerate}
		\item $ L = \{ 0^i 1^j 2^i 3^j |i,j \ge 1\} $
		\item $ L = \{ a^i b^j c^k |0 \le i < j < k\} $
		\item $ L = \{ a^i b^j |j = i^2 \} $
		\item $ {\bar L}$, where $L = \{ 0^k |k$  is a perfect square $\}$
	\end{enumerate}

	\item Design a PDA and provide a context-free grammar (in any form) to accept the following language: 
	\begin{enumerate}
	\item $L = \{ a^n b^{n + m} c^m |n \ge 0,m \ge 1\} $
	\item The set of all strings over $\{a, b\}$ with exactly twice as many $a$'s as $b$'s.
	\end{enumerate}
	
	\item Give a context-free grammar in Chomsky Normal Form that generates the following language: 
	\begin{enumerate}
	\item The set of all strings over $\{a, b\}$ with exactly twice as many $a$'s as $b$'s.
	\item $L = \{ w \in (a + b + c)^* |n_a (w) + n_b (w) \ne n_c (w)\}$, where  $n_a (w)$ is the number of $a$'s in $w$.\\	
	\item $L = \{ a^i b^j c^k |i \ne j$  or $j \ne k\} $
	\end{enumerate}
	
\end{enumerate}

\end{document}
