%% Based on a TeXnicCenter-Template by Gyorgy SZEIDL.
%%%%%%%%%%%%%%%%%%%%%%%%%%%%%%%%%%%%%%%%%%%%%%%%%%%%%%%%%%%%%

%------------------------------------------------------------
%
\documentclass{article}%
%Options -- Point size:  10pt (default), 11pt, 12pt
%        -- Paper size:  letterpaper (default), a4paper, a5paper, b5paper
%                        legalpaper, executivepaper
%        -- Orientation  (portrait is the default)
%                        landscape
%        -- Print size:  oneside (default), twoside
%        -- Quality      final(default), draft
%        -- Title page   notitlepage, titlepage(default)
%        -- Columns      onecolumn(default), twocolumn
%        -- Equation numbering (equation numbers on the right is the default)
%                        leqno
%        -- Displayed equations (centered is the default)
%                        fleqn (equations start at the same distance from the right side)
%        -- Open bibliography style (closed is the default)
%                        openbib
% For instance the command
%           \documentclass[a4paper,12pt,leqno]{article}
% ensures that the paper size is a4, the fonts are typeset at the size 12p
% and the equation numbers are on the left side
%
\usepackage{amsmath}%
\usepackage{amsfonts}%
\usepackage{amssymb}%
\usepackage{graphicx}
%-------------------------------------------
\newtheorem{theorem}{Theorem}
\newtheorem{acknowledgement}[theorem]{Acknowledgement}
\newtheorem{algorithm}[theorem]{Algorithm}
\newtheorem{axiom}[theorem]{Axiom}
\newtheorem{case}[theorem]{Case}
\newtheorem{claim}[theorem]{Claim}
\newtheorem{conclusion}[theorem]{Conclusion}
\newtheorem{condition}[theorem]{Condition}
\newtheorem{conjecture}[theorem]{Conjecture}
\newtheorem{corollary}[theorem]{Corollary}
\newtheorem{criterion}[theorem]{Criterion}
\newtheorem{definition}[theorem]{Definition}
\newtheorem{example}[theorem]{Example}
\newtheorem{exercise}[theorem]{Exercise}
\newtheorem{lemma}[theorem]{Lemma}
\newtheorem{notation}[theorem]{Notation}
\newtheorem{problem}[theorem]{Problem}
\newtheorem{proposition}[theorem]{Proposition}
\newtheorem{remark}[theorem]{Remark}
\newtheorem{solution}[theorem]{Solution}
\newtheorem{summary}[theorem]{Summary}
\newenvironment{proof}[1][Proof]{\textbf{#1.} }{\ \rule{0.5em}{0.5em}}


\usepackage{color, ulem}
\definecolor{dkgreen}{rgb}{0, 0.75, 0}
\definecolor{dkred}{rgb}{0.5, 0, 0}
\definecolor{dkpurp}{rgb}{0.25, 0, 0.5}
\newcommand{\add}[1]{\textcolor{dkgreen}{#1}}
\newcommand{\rmv}[1]{\textcolor{red}{\sout{#1}}}
\newcommand{\moveto}[1]{\textcolor{blue}{#1}}
\newcommand{\movefrom}[1]{\textcolor{blue}{\sout{#1}}}
\newcommand{\highlighttext}[1]{\colorbox{yellow}{#1}}
\newenvironment{added}{\color{dkgreen}}{\color{black}}
\newenvironment{removed}{\color{red}}{\color{black}}
\newenvironment{edited}{\color{blue}}{\color{black}}

\begin{document}

\begin{flushleft}
\textbf{Course:} CSC707, Automata, Computability and Computational Theory\\
\textbf{Homework 3}: Self-reduction and Approximability. \\
\textbf{Submission:} Use Wolfware\\
\textbf{File Format:} LaTeX and PDF\\
\end{flushleft}

\begin{center}
\fbox{\textbf{Due Date:} \textbf{2:00 A.M. (EST), Thursday, February 18, 2010}}\\
\begin{enumerate}
	\item Provide any feedback/questions you may have on this homework (\textbf{optional}).
	\item Using LaTeX is required.
\end{enumerate}\end{center}

\noindent{\hrulefill}

\bigskip

	\add{	My points: 8.25 out of 10}

\begin{enumerate}

%---------- PROBLEM 4 --------------
	\item \textbf{Approximability}: 
	\begin{enumerate}
	\item  If $P \neq NP$, then for any constant $c \geq 1$, there is no polynomial-time $c$-approximation algorithm for a general Traveling Salesman Problem. (Hint: Show that HAM-CYCLE can be solved in polynomial time. Also, see the additional hint in the lecture slides.)
	
		\end{enumerate}
		
		\add{
		\begin{theorem}
		If $P \neq NP$, then for any constant $c \geq 1$, there is no polynomial-time $c$-approximation algorithm for a general Traveling Salesman Problem.
		\end{theorem}	
		-0.5 Pt: I must clearly state the theorem I want to prove.	
		}		
		
		\rmv{
		\textbf{Solution:}
		}	
		\begin{proof}
\begin{enumerate}
	\item 
\begin{enumerate}
	\item We first prove that HAM-CYCLE problem can be solved by using the $c$-approximation algorithm for a general Traveling Salesman Problem. Given a undirected and unweighted graph $G$, we can create a complete graph $G'$ of $G$ by adding missing edges and assign weight to each edge. For the edges in $G$, we assign 1 for the weight. For the edges not in $G$, we assign $c*|V| + 1$ for the weight. This reduction can be done in polynomial time ($O(|V|*|E|)$). \add{(-1 Pt: Since $|E|=O(|V|^2)$, the reduction is $O(n^3)$)} Using the $c$-approximation algorithm for TSP, we can get a TSP tour of weight $W$. If $W > c*|V|$, then we know this tour pick at least one edge that is not in $G$ (since every edge in $G$ is assigned weight of 1) and $G$ does not have HAM-CYCLE. Otherwise, $G$ has a HAM-CYCLE.
  \item  Since the reduction from HAM-CYCLE to TSP is in polynomial time, if there exists a polynomial-time $c$-approximation algorithm for a general Traveling Salesman Problem, HAM-CYCLE can be solved in polynomial time. We already know that \rmv{HAM-CYCLE $\in NP$} \add{-0.25 Pt: HAM-CYCLE is $NP-complete$ problem}. If it can be solved in polynomial time, then $P=NP$, which contradicts with \rmv{out} \add{our} assumption that $P \neq NP$. Thus, there is no polynomial-time $c$-approximation algorithm for a general Traveling Salesman Problem. 
\end{enumerate}
\end{enumerate}
		\end{proof}
		
		\end{enumerate}
\end{document}
