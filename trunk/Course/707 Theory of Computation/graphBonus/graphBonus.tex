\documentclass[article, 10pt,onecolumn]{article} 
\usepackage{latex8}
\usepackage{times}
\usepackage{url}
%\usepackage{amsfonts, amsthm}

\begin{document}
\begin{flushleft}
\textbf{Course:} CSC707, Automata, Computability and Computational Theory\\
\textbf{Bonus Homework}: Bonus points assignment\\
\textbf{Author}: Xusheng Xiao(\small{xxiao2@ncsu.edu}), Xi Ge(\small{xge@ncsu.edu}), Da Young Lee(\small{dlee10@ncsu.edu})\\
\end{flushleft}

\begin{center}
\fbox{\textbf{Due Date:} \textbf{2:00 AM, Monday, Feb 2nd, 2010}}\\
\end{center}
\noindent{\hrulefill}
\begin{enumerate}
\item \textbf{Clique:}
	\begin{enumerate}
	\item\textbf{Description of real-world problem:}\\ 
   we study the people in a party to find out groups that every two persons in a group know each other. 
  \item \textbf{Identification of the corresponding graph problem:}\\
  $CKIQUE$=$\{\langle $G,k $ \rangle | $ a graph $G$ has a clique of size $k\}$ 
  \item \textbf{Description of the reduction from real to graph problem:}
  	\begin{enumerate}
		\item represent each person as a vertex in graph.
		\item if two people know each other create an edge between them.
		\item the group that each two persons know each other could be reduced to the clique in this graph.
		\end{enumerate}
	\end{enumerate}	  
\item \textbf{Vertex Cover:}
  \begin{enumerate}
   \item \textbf{Description of real-world problem:}\\
   We study the group of people in a party to find people who
   know all other people in the party. The minimum vertex cover problem
   is to find the minimum number of people who know all other people.
   
      \item \textbf{Identification of the corresponding graph problem:}\\
    $VC$ = $\{\langle $G,k $ \rangle | $ a graph $G$ has a vertex cover of size $k\}$. 
    Given a graph, a vertex cover is a set $S$ of vertices and every edge (of the graph) should be incident to at least one vertex $v$ ($v$ $\in$ $S$).
   \item \textbf{Description of the reduction from real to graph problem:}\\
       The steps in resolving the vertex cover problem to find people who know all other people includes: 
    \begin{enumerate}
	    \item Specify vertices, each of which represents an individual person.
	    \item Specify edges, each of which represents relation of two persons such that one person knows another person.
	    \item Given vertices and edges, draw a relation diagram.
	    \item Given a graph, resolve the vertex cover problem to find a set of people who know all the people in the party.
    \end{enumerate}
    Given the preceding example, the vertex cover problem may help a party organizer
    find people who know all the other people.
  \end{enumerate}
  
\item \textbf{Independent Set:}
\begin{enumerate}
 		\item\textbf{Description of real-world problem:} \\
   		We study the people in the party to find out group that no two people in it know each other.
  	\item\textbf{Identification of the corresponding graph problem:}\\
  		$IS$=$\{\langle $G,k $ \rangle | $ a graph $G$ has an independent set of size $k\}$
  	\item \textbf{Description of the reduction from real to graph problem:}	
  		\begin{enumerate}
			\item represent each person as a vertex.
			\item if two people know each other, draw a edge between them.
			\item finding the group that no two people know each other could be reduced to finding the independent set of this graph.
			\end{enumerate}
  	
\end{enumerate}
\item \textbf{Dominating Set:} 
	\begin{enumerate}
		\item\textbf{Description of real-world problem:}\\ 
   		We study the group of people in a party that every people in the party knows at least one people in the group. 
  	\item\textbf{Identification of the corresponding graph problem:}\\
  		$DS$=$\{\langle $G,k $ \rangle | $ a graph $G$ has a dominating set of size $k\}$
  	\item
  		\begin{enumerate}
				\item represent each person as a vertex in a graph.
				\item if two people know each other, creat an edge between these two nodes.
				\item finding the group that everyone in the party knows at least one people in the group could be reduced to finding the dominating set of the graph.
			\end{enumerate}
	\end{enumerate}
\item \textbf{Hamiltonian Path:} 
\begin{enumerate}
	\item \textbf{Description of real-world problem:}\\
    To tour the 10 biggest cities in the UK, starting in London (number 1) and finishing in Bristol (number 10)is an example of finding a
    Hamilton Path.~\cite{path}
   \item \textbf{Identification of the corresponding graph problem:}\\
    A Hamiltonian path (or traceable path) is a path in an undirected graph which visits each vertex exactly once. 
   \item \textbf{Description of the reduction from real to graph problem:}\\
   We can draw a graph like this: each city represents a vertex and each road connecting each two cities reprents an edge. Find a tour to visit all the cities and finish at the last city is to find out a Hamiltonian Path that visit every vertex.
 \end{enumerate}  
\item \textbf{Hamiltonian Cycle:} 
\begin{enumerate}
	\item  \textbf{Description of real-world problem:}\\
    The `Knight's Tour'~\cite{knight} is a sequence of moves done by a knight on a chessboard. The knight is placed on an empty chessboard
    and, following the rules of chess, must visit each square exactly once. The Knight's Tour problem is an instance of the more general
    Hamiltonian path problem in graph theory. This problem of getting a closed Knight's Tour is similarly an instance of the Hamiltonian
    cycle problem.
   \item \textbf{Identification of the corresponding graph problem:}\\
    A Hamiltonian cycle (or Hamiltonian circuit) is a cycle in an undirected graph which visits each vertex exactly once and also returns to the starting vertex.
   \item \textbf{Description of the reduction from real to graph problem:}\\
    Give a chess borad, we can draw a graph from the path of a Knight's Tour: each vertex of the graph represents a square of the board and each edge represents a knight's move. The way to visit each square and have the knight finishing on a square which is just a move away from the starting square create a tour that is described as 're-entrant' or 'closed'. Such a tour is a Hamiltonian Cycle.
  \end{enumerate}
  
  
\item \textbf{Shortest Path:} 
\begin{enumerate}   
   \item \textbf{Description of real-world problem:}\\
    The shortest path problem is often used to find the shortest path between two locations (e.g., cities).
    Consider a map includes multiple cities and some of these cities are connected with roads.
    Given two cities over the map, there could be many paths between the two cites.
    A navigation system calculates which path is the shortest path (among all the possible paths) to help
    guide the best path in terms of length.    

   \item \textbf{Identification of the corresponding graph problem:}\\
    To find the shortest path is to find a path least with
    total weight (of edges) from one vertex to another vertex in a given graph.
   \item \textbf{Description of the reduction from real to graph problem:}\\
    The steps in resolving the shortest path problem between two locations (e.g., cities) includes: 
    \begin{enumerate}
	    \item Specify vertices representing locations.
	    \item Specify edges with weights, each of which represents length of roads of connected locations (i.e., vertices).
	    \item Given vertices and edges, draw a location diagram.
	    \item Given two locations, find the shortest path of the two locations over a location diagram.
    \end{enumerate}
    The shortest path is the shortest-length path through the diagram. The significance of the shortest path problem is that a user
    can move from one location to another location through the shorted path, which is a path with least length of roads.
   
\end{enumerate}
 
\item \textbf{Longest Path:} 
 \begin{enumerate}
   \item \textbf{Description of real-world problem:}\\
    Longest path is used to identify critical path in Critical Path Method~\cite{cpm}, a project
    management method. It is to address the challenge of shutting down chemical plants for maintainance and then restarting the plants once
    the maintainance had been completed. 
   \item \textbf{Identification of the corresponding graph problem:}\\
    The longest path problem is the problem of finding a simple path of maximum length in a given graph.
   \item \textbf{Description of the reduction from real to graph problem:}\\
    The steps in CPM project planning includes: 
    \begin{enumerate}
	    \item Specify the individual activities.
	    \item Determine the sequence of those activities.
	    \item Draw a network diagram.
	    \item Estimate the completion time for each activity.
	    \item Identify the critical path (\textbf{longest path through the network})
	    \item Update the CPM diagram as the project progresses.
    \end{enumerate}
    The critical path is the longest-duration path through the network. The significance of the critical path is that the activities lie on
    it cannot be delayed without delaying the project.
 \end{enumerate}
\end{enumerate}
\bibliographystyle{plain}
\bibliography{references}
\end{document}

%\title{Graph Bonus}
%\author{
%Xusheng Xiao\\
%\small{xxiao2@ncsu.edu}\\
%\and
%Xi Ge\\
%\small{xge@ncsu.edu}\\
%\and
%Da Young Lee\\
%\small{dlee10@ncsu.edu}
%}
%\date{January 27, 2010}
%\maketitle
%\newcommand{\N}{\mathbb{N}}
%\newcommand{\Z}{\mathbb{Z}}
%\newcommand{\R}{\mathbb{R}}
%\begin{document}
%\maketitle

%\begin{flushleft}
%\textbf{Hamilton Path}:\end{flushleft} To tour the 10 biggest cities in the UK, starting in London (number 1) and finishing in Bristol (number 10)is an example of finding a Hamilton Path.~\cite{path}
%
%\begin{flushleft}
%\textbf{Hamilton Cycle}:\end{flushleft} The ``Knight's Tour''~\cite{knight} is a sequence of moves done by a knight on a chessboard. The knight is placed on an empty chessboard and, following the rules of chess, must visit each square exactly once. The Knight's Tour problem is an instance of the more general Hamiltonian path problem in graph theory. This problem of getting a closed Knight's Tour is similarly an instance of the Hamiltonian cycle problem.
