%% Based on a TeXnicCenter-Template by Gyorgy SZEIDL.
%%%%%%%%%%%%%%%%%%%%%%%%%%%%%%%%%%%%%%%%%%%%%%%%%%%%%%%%%%%%%

%------------------------------------------------------------
%
\documentclass{article}%
%Options -- Point size:  10pt (default), 11pt, 12pt
%        -- Paper size:  letterpaper (default), a4paper, a5paper, b5paper
%                        legalpaper, executivepaper
%        -- Orientation  (portrait is the default)
%                        landscape
%        -- Print size:  oneside (default), twoside
%        -- Quality      final(default), draft
%        -- Title page   notitlepage, titlepage(default)
%        -- Columns      onecolumn(default), twocolumn
%        -- Equation numbering (equation numbers on the right is the default)
%                        leqno
%        -- Displayed equations (centered is the default)
%                        fleqn (equations start at the same distance from the right side)
%        -- Open bibliography style (closed is the default)
%                        openbib
% For instance the command
%           \documentclass[a4paper,12pt,leqno]{article}
% ensures that the paper size is a4, the fonts are typeset at the size 12p
% and the equation numbers are on the left side
%
\usepackage{amsmath}%
\usepackage{amsfonts}%
\usepackage{amssymb}%
\usepackage{graphicx}
%-------------------------------------------
\newtheorem{theorem}{Theorem}
\newtheorem{acknowledgement}[theorem]{Acknowledgement}
\newtheorem{algorithm}[theorem]{Algorithm}
\newtheorem{axiom}[theorem]{Axiom}
\newtheorem{case}[theorem]{Case}
\newtheorem{claim}[theorem]{Claim}
\newtheorem{conclusion}[theorem]{Conclusion}
\newtheorem{condition}[theorem]{Condition}
\newtheorem{conjecture}[theorem]{Conjecture}
\newtheorem{corollary}[theorem]{Corollary}
\newtheorem{criterion}[theorem]{Criterion}
\newtheorem{definition}[theorem]{Definition}
\newtheorem{example}[theorem]{Example}
\newtheorem{exercise}[theorem]{Exercise}
\newtheorem{lemma}[theorem]{Lemma}
\newtheorem{notation}[theorem]{Notation}
\newtheorem{problem}[theorem]{Problem}
\newtheorem{proposition}[theorem]{Proposition}
\newtheorem{remark}[theorem]{Remark}
\newtheorem{solution}[theorem]{Solution}
\newtheorem{summary}[theorem]{Summary}
\newenvironment{proof}[1][Proof]{\textbf{#1.} }{\ \rule{0.5em}{0.5em}}
\usepackage{color, ulem}
\definecolor{dkgreen}{rgb}{0, 0.75, 0}
\definecolor{dkred}{rgb}{0.5, 0, 0}
\definecolor{dkpurp}{rgb}{0.25, 0, 0.5}
\newcommand{\add}[1]{\textcolor{dkgreen}{#1}}
\newcommand{\rmv}[1]{\textcolor{red}{\sout{#1}}}
\newcommand{\moveto}[1]{\textcolor{blue}{#1}}
\newcommand{\movefrom}[1]{\textcolor{blue}{\sout{#1}}}
\newcommand{\highlighttext}[1]{\colorbox{yellow}{#1}}
\newenvironment{added}{\color{dkgreen}}{\color{black}}
\newenvironment{removed}{\color{red}}{\color{black}}
\newenvironment{edited}{\color{blue}}{\color{black}}
\begin{document}

\begin{flushleft}
\textbf{Course:} CSC707, Automata, Computability and Computational Theory\\
\textbf{Group Problem}: Clique $\propto$ Independent Set. \\
\textbf{Students:} kshakya, dsen, dbao\\
\textbf{Graded by:} Xusheng Xiao, Ling Chen, Xi Ge\\
\textbf{Minus Points:} \textcolor{red}{-19}
\end{flushleft}


\noindent{\hrulefill}

\bigskip


\begin{flushleft}
\textbf{Input}: A graph G =(V,E), an integer k $\leq \left | V \right |$ \\
\textbf{Output}: If there is a subset S of k vertices in G such that no pair of vertices in S is connected by an edge in G?\\

\textbf{Decision Problem}: \\
IS =  $\left \{ <  G,k > : \textup{Graph G has a Independent set of size k } \right \} $\\

\textbf{Step 1}:\\
We need to show that Independent Set (IS) $ \in $ NP. Suppose we have been given a set of 
vertices $V'$. We can verify that $V'$ is IS of Graph G as follows:\\
\begin{enumerate}
	\item Verify $\left | V' \right | $ = k. This is O(1).
	\item For each edge (u,v) $\in$ E(G), we check whether u $\in$ $V'$ and v$\in$ $V'$. If none of the edge satisfies this relation, then $V'$ is IS of G, otherwise it is not. This is $O(k^2\left |E\right |)$.
\end{enumerate}

So the verification step is polynomial. Hence IS is in NP.

\textbf{Step2}:\\
Given the Clique problem is NP-Complete, we show that Clique $\propto$ IS. Given an instance of Clique problem $<G,k>$, following steps reduces it to IS problem.\\
\begin{enumerate}
\item We compute the complement $\overline{G}$ of given Graph. This take \rmv{$O(n^2)$}\add{$O(|V|^2)$}. \textcolor{red}{-2}
\item Calculate $k'=k$.
\end{enumerate}
The output of reduction algorithm is an instance $<\overline{G},k'>$ of IS problem.

\textbf{Step3}:\\
\rmv{Graph G has Clique $V' \subset$ V}\add{$G$ has a clique of size $k$} iff \rmv{$\overline{G}$ has IS $V'$}\add{$\overline{G}$ has IS of size $k$}.  \textcolor{red}{-1} \\
$\rightarrow$\\
\rmv{
Suppose G has Clique $V' \subset$ V. If (u,v) be an edge in G and u $\in$ $V'$ and v $\in$ $V'$, then by definition of Clique and Complement, (u,v) $\notin \overline{E}$. Since u,v were chosen arbitrarily, the vertices that forms a clique in G forms IS in $\overline{G}$. \textcolor{red}{-8}\\
}
\add{Consider any two vertices $u,v\in Clique(G)$, suppose that edge $(u,v)\in \overline{G}$. By the definition of complement graph, $(u,v)\notin G$, which goes contradict with that $u,v\in Clique(G)$. Therefore for $\forall u,v\in Clique(G)$, $(u,v)\notin \overline E$. $\overline G$ has a IS of size k.
}\\
$\leftarrow$\\

\rmv{
Suppose G has IS $V' \subset$ V.  Take any two verices u,v in G such that u $\in$ $V'$ and v $\in$ $V'$, then (u,v) $\notin {E}$ by defition of Independent Set. This implies that (u,v)  $\in \overline{E}$.  Since u,v were chosen arbitrarily, the vertices that forms a IS in G forms a Clique of same size in $\overline{G}$. \textcolor{red}{-8}\\
}
\add{
Suppose $\overline{G}$ has an IS $V' \subset V, |V'|=k$, $u \in V', v \in V'$, $\Rightarrow$ $(u,v) \notin \overline{E}$. By the definition of complement graph, $(u,v)\in E$. Therefore, $V'$ is an Clique of $G$ $\Rightarrow$ $G$ has a clique of size $k$.}\\
This proves that IS is NP-Complete problem.


\end{flushleft}
\end{document}
