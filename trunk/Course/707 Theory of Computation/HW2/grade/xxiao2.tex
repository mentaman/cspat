%% Based on a TeXnicCenter-Template by Gyorgy SZEIDL.
%%%%%%%%%%%%%%%%%%%%%%%%%%%%%%%%%%%%%%%%%%%%%%%%%%%%%%%%%%%%%

%------------------------------------------------------------
%
\documentclass{article}%
%Options -- Point size:  10pt (default), 11pt, 12pt
%        -- Paper size:  letterpaper (default), a4paper, a5paper, b5paper
%                        legalpaper, executivepaper
%        -- Orientation  (portrait is the default)
%                        landscape
%        -- Print size:  oneside (default), twoside
%        -- Quality      final(default), draft
%        -- Title page   notitlepage, titlepage(default)
%        -- Columns      onecolumn(default), twocolumn
%        -- Equation numbering (equation numbers on the right is the default)
%                        leqno
%        -- Displayed equations (centered is the default)
%                        fleqn (equations start at the same distance from the right side)
%        -- Open bibliography style (closed is the default)
%                        openbib
% For instance the command
%           \documentclass[a4paper,12pt,leqno]{article}
% ensures that the paper size is a4, the fonts are typeset at the size 12p
% and the equation numbers are on the left side
%
\usepackage{amsmath}%
\usepackage{amsfonts}%
\usepackage{amssymb}%
\usepackage{graphicx}
%-------------------------------------------
\newtheorem{theorem}{Theorem}
\newtheorem{acknowledgement}[theorem]{Acknowledgement}
\newtheorem{algorithm}[theorem]{Algorithm}
\newtheorem{axiom}[theorem]{Axiom}
\newtheorem{case}[theorem]{Case}
\newtheorem{claim}[theorem]{Claim}
\newtheorem{conclusion}[theorem]{Conclusion}
\newtheorem{condition}[theorem]{Condition}
\newtheorem{conjecture}[theorem]{Conjecture}
\newtheorem{corollary}[theorem]{Corollary}
\newtheorem{criterion}[theorem]{Criterion}
\newtheorem{definition}[theorem]{Definition}
\newtheorem{example}[theorem]{Example}
\newtheorem{exercise}[theorem]{Exercise}
\newtheorem{lemma}[theorem]{Lemma}
\newtheorem{notation}[theorem]{Notation}
\newtheorem{problem}[theorem]{Problem}
\newtheorem{proposition}[theorem]{Proposition}
\newtheorem{remark}[theorem]{Remark}
\newtheorem{solution}[theorem]{Solution}
\newtheorem{summary}[theorem]{Summary}
\newenvironment{proof}[1][Proof]{\textbf{#1.} }{\ \rule{0.5em}{0.5em}}

\usepackage{color, ulem}
\definecolor{dkgreen}{rgb}{0, 0.75, 0}
\definecolor{dkred}{rgb}{0.5, 0, 0}
\definecolor{dkpurp}{rgb}{0.25, 0, 0.5}
\newcommand{\add}[1]{\textcolor{dkgreen}{#1}}
\newcommand{\rmv}[1]{\textcolor{red}{\sout{#1}}}
\newcommand{\moveto}[1]{\textcolor{blue}{#1}}
\newcommand{\movefrom}[1]{\textcolor{blue}{\sout{#1}}}
\newcommand{\highlighttext}[1]{\colorbox{yellow}{#1}}
\newenvironment{added}{\color{dkgreen}}{\color{black}}
\newenvironment{removed}{\color{red}}{\color{black}}
\newenvironment{edited}{\color{blue}}{\color{black}}

\begin{document}

\begin{flushleft}
\textbf{Course:} CSC707, Automata, Computability and Computational Theory\\
\textbf{Homework 2}: Complexity Theory, polynomial time reduction, P vs NP, NP-hard and NP-complete problems. \\
\textbf{Submission:} Use Wolfware\\
\textbf{File Format:} LaTeX and PDF\\
\end{flushleft}

\begin{center}
\fbox{\textbf{Due Date:} \textbf{2:00 A.M. (EST), Thursday, February 11, 2010}}\\
\begin{enumerate}
	\item Provide any feedback/questions you may have on this homework (\textbf{optional}).
	\item Using LaTeX is required.
\end{enumerate}\end{center}

\noindent{\hrulefill}

\bigskip

\add{Your Points: 6 out os 10.}

\begin{enumerate}


  \item Show that the Vertex Cover remains $NP$-complete even when all the vertices in the graph 	are restricted to have even degree.\\
  Vertex Cover is defined as follows:\\
  INSTANCE: A graph $G=(V,E)$ and a positive integer $k \leq |V|$.\\
  QUESTION: Is there a subset $V' \subseteq V$ such that $|V'| \leq k$, and for each edge $\{u,v\} \in E$ at least one of $u$ and $v$ belongs to $V'$?
  
  \add{
		\begin{theorem}
		$VC_{Even}$ is $NP$-complete.
		\end{theorem}
		}
		
		\add{-1 Pt: You must clearly state your theorem that you plan to prove. Also, see how I added begin and end proof statements around your proof.}
		
\begin{proof}
  
  \textbf{Proof}:\\
  \emph{Decisition Problem}: $VC=\{<G,k>:$ graph $G$ has a vertex cover of size $k$ where $\forall e \in E, degree(e) = 2\}$
  
  \add{-0.5 Pt: $degree(e)$ is undefined and is not 2. Even degree vertex?}
  
  \emph{Step 1}:\\
  Given a vertex cover of size $k$ of a graph $G=(V,E),\forall v \in V degree(v) \text{is even}$, we can verify that it is indeed the vertex cover of graph $G(n,k)$ in polynomial time:\\
  $$T(n,k)=O(|E|k)=O(n^{2}k)$$
  
  \add{-1 Pt: How will you verify? In terms of $n$, what is the complexity?}
  
   \emph{Step 2}:\\
  Given a $G=(V,E)$, we can construct a graph $G'=(V',E')$ such that:\\
  $V' = V \bigcup \{v_{1},v_{2},v_{3}\}$, and \\
  $E' = E \bigcup \{<v_{1},v_{2}>,<v_{1},v_{3}>,<v_{2},v_{3}>\}$ $\bigcup \{<v,v_{1}>|v \in V, degree(v) \text{is odd}>\}$ \\
  
  We claim that all the vertices in $G'$ have even degree since the number of vertices that has odd degree is even. \\
  
 
  
  \textbf{Proof}:\\ 
  The sum of the degree of all vertices in a graph $G$ with $n$ edges is $2n$ since each edge has two end points and add 1 to the degree of two end points(vertices). Assume the set of odd-degree vertices is $V_{o}$ and the set of even-degree vertices is $V_{e}$, then, we have
  $$\sum_{v\in V}degree(v) = \sum_{v'\in V_{o}}degree(v') +  \sum_{v''\in V_{e}}degree(v'')$$ \\
  $\sum_{v\in V}degree(v)=2n$ is even and $ \sum_{v''\in V_{e}}degree(v'')$ is even (sum of even numbers is even) $\Rightarrow$
  $\sum_{v'\in V_{o}}degree(v')$ is even $\Rightarrow$ the number of odd number must be even\\
  
  Thus, all the vertices of $G'$ has even degrees. 
  
  The construction of $G'$ can be completed in polynomial time since we only need to visit every vertex in $G$ for one time:
  $$T(n,k)=O(|E|)=O(n)$$
  
  \add{-1 Pt: Time complexity is wrong.}
  
  \emph{Step 3}:\\
  \textbf{Claim}: The graph $G$ has a vertex cover of size $k$ if and only if $G'$ has a vertex cover of size $k+2$\\
  Suppose graph $G=(V,E)$ has vertex cover of size $k$, $V' \subseteq V:|V'|=k$, and we choose a set of vertices $V'' = V' \bigcup \{v_{1},v_{2}\}$:
  
\begin{enumerate}
	\item $v_{1} \in V''$ $\Rightarrow$ edges $e' \in \{<v,v_{1}>|v \in V, degree(v) is odd>\}$ is covered by $v_{1}$. 
	\item $v_{1} \in V'',v_{2} \in V''$ $\Rightarrow$ $<v_{1},v_{2}>,<v_{1},v_{3}>,<v_{2},v_{3}>$ is covered by $v_{1},v_{2}$. 
	\item $\forall e \in E$ is covered by $V', V' \subseteq V''.$ $\Rightarrow$ $V''$ cover all the edges in $E$ 
\end{enumerate}
  
  Combining (a),(b), and (c) $\Rightarrow$ $V''$ is the vertex cover of $G'$.\\
 
  Suppose set $G'$ has a set cover $V''$ of size $k+2$\\
  To cover $<v_{1},v_{2}>,<v_{1},v_{3}>,<v_{2},v_{3}>$, at least two vertices of $\{v_{1},v_{2},v_{3}\}$ are included in $V''$ \add{(-0.5 Pt: Why?)}\\
  Since all egdes in the $E'$ are covered by $V''$, the edges except $<v_{1},v_{2}>,<v_{1},v_{3}>,<v_{2},v_{3}>$ must be covered by $k$ vertices. These edges are exactly the same edges in $E$.\\
  Thus, $G$ has a vertex cover of $k$. 
  
  \end{proof}
  
\end{enumerate}

\end{document}
