%% Based on a TeXnicCenter-Template by Gyorgy SZEIDL.
%%%%%%%%%%%%%%%%%%%%%%%%%%%%%%%%%%%%%%%%%%%%%%%%%%%%%%%%%%%%%

%------------------------------------------------------------
%
\documentclass{article}%
%Options -- Point size:  10pt (default), 11pt, 12pt
%        -- Paper size:  letterpaper (default), a4paper, a5paper, b5paper
%                        legalpaper, executivepaper
%        -- Orientation  (portrait is the default)
%                        landscape
%        -- Print size:  oneside (default), twoside
%        -- Quality      final(default), draft
%        -- Title page   notitlepage, titlepage(default)
%        -- Columns      onecolumn(default), twocolumn
%        -- Equation numbering (equation numbers on the right is the default)
%                        leqno
%        -- Displayed equations (centered is the default)
%                        fleqn (equations start at the same distance from the right side)
%        -- Open bibliography style (closed is the default)
%                        openbib
% For instance the command
%           \documentclass[a4paper,12pt,leqno]{article}
% ensures that the paper size is a4, the fonts are typeset at the size 12p
% and the equation numbers are on the left side
%
\usepackage{amsmath}%
\usepackage{amsfonts}%
\usepackage{amssymb}%
\usepackage{graphicx}
%-------------------------------------------
\newtheorem{theorem}{Theorem}
\newtheorem{acknowledgement}[theorem]{Acknowledgement}
\newtheorem{algorithm}[theorem]{Algorithm}
\newtheorem{axiom}[theorem]{Axiom}
\newtheorem{case}[theorem]{Case}
\newtheorem{claim}[theorem]{Claim}
\newtheorem{conclusion}[theorem]{Conclusion}
\newtheorem{condition}[theorem]{Condition}
\newtheorem{conjecture}[theorem]{Conjecture}
\newtheorem{corollary}[theorem]{Corollary}
\newtheorem{criterion}[theorem]{Criterion}
\newtheorem{definition}[theorem]{Definition}
\newtheorem{example}[theorem]{Example}
\newtheorem{exercise}[theorem]{Exercise}
\newtheorem{lemma}[theorem]{Lemma}
\newtheorem{notation}[theorem]{Notation}
\newtheorem{problem}[theorem]{Problem}
\newtheorem{proposition}[theorem]{Proposition}
\newtheorem{remark}[theorem]{Remark}
\newtheorem{solution}[theorem]{Solution}
\newtheorem{summary}[theorem]{Summary}
\newenvironment{proof}[1][Proof]{\textbf{#1.} }{\ \rule{0.5em}{0.5em}}

\begin{document}

\begin{flushleft}
\textbf{Course:} CSC707, Automata, Computability and Computational Theory\\
\textbf{EXAM 1}: Countability, closure properties of countable sets. Complexity theory, NP-completeness, polynomial-time reducibility, self-reduction, approximability.\\
\textbf{Submission:} Home-take\\
%\textbf{Duration:} 75 minutes\\
\textbf{FULL NAME:} \rule{2 in}{1 pt} \\
\end{flushleft}

\begin{center}
\fbox{\textbf{Due Date: February 25, 11:59 PM}
\textbf{Mid-Term Exam, Spring, 2010}}\\
\end{center}

\noindent{\hrulefill}

\begin{enumerate}
	\item Provide a solution to \textbf{ONE} problem in \textbf{EACH} category; a total of \textbf{FOUR} categories.
	\item Solutions to the other problems in each category will\textbf{ NOT} be graded. 
	\item This is an open-textbook (main course book), open-notes, open-homeworks, BUT \textbf{closed-internet} exam. You may \textbf{NOT} discuss the exam with any one. The exam is an \textbf{INDIVIDUAL} effort.
\end{enumerate}
\noindent{\hrulefill}


\bigskip

\begin{enumerate}

	%---------- PROBLEM 1 --------------
	\item Prove or disprove the countability of each of the following sets (\textbf{25 points}: identifying which class (5 points), proof idea (5 points), complete proof (15 points)):	
	\begin{enumerate}
		\item The set of all regular languages over $\{0, 1\}$.
		\item The set of all languages over the alphabet $\{0, 1\}$
		\item The set of all infinite length strings over the three letter alphabet $\{0, 1, 2\}$
	\end{enumerate}
	
		
	%---------- PROBLEM 2 --------------
	\item \textbf{$NP$-completeness, 25 points}: Verification Step (5 points), Reduction Step (10 points), Correctness Step (10 points): Prove that the problem is $NP$-complete.
	
	\begin{enumerate}
		\item BIGGER-CLIQUE = $\{\langle G_1, G_2\rangle \mid $ the largest clique of graph $G_1$ is larger than every clique of graph $G_2\}$
		\item NP-PATH = {$<G, s, t, k>$ | $G$ is an undirected graph containing a simple path of length at least $k$ from $s$ to $t$.} (Hint: \textit{Use the fact that the Hamiltonian Path problem for undirected graphs is $NP$-complete}.)
		\item HALF-CLIQUE = {$<G>$ | $G$ is an undirected graph having a complete subgraph with �$\left\lfloor n/2 \right\rfloor$ nodes, where $n$ is the number of nodes in $G$}.
		%\item Longest Path (Reduction from Hamiltonian Path)
		%\item MAX-CUT (See hint for Problem 7.25 in Sipser's book)
	\end{enumerate}

%---------- PROBLEM 3 --------------

	\item \textbf{Self-reducibility, 25 points}: 
	\begin{enumerate}
		\item Find an isomorphism between graphs $G_1$ and $G_2$ and provide time complexity, or state that none exists.  (An isomorphism is a bijection $\phi : V(G_1) \to V(G_2)$ such that $(v_1, v_2) \in E(G_1)$ iff $(\phi(v_1), \phi(v_2)) \in E(G_2)$.)  Assume that you have a decision algorithm $D(G, G')$ that decides whether $G$ and $G'$ are isomorphic in $O(f(|V(G)| + |V(G')|))$ time.
		
		\item Given a set of integers $A$, find a subset of $A$ that sums to zero and provide time complexity, or state that none exists.  Assume that you have a decision algorithm $D(S)$ that decides whether such a subset exists in $O(f(|S|))$ time.	
	\end{enumerate}

%---------- PROBLEM 4 --------------
	
	\item \textbf{Approximability, 25 points}: 
	\begin{enumerate}
	\item  If $P \neq NP$, then Vertex Cover does not allow any absolute approximation. (There is an absolute approximation algorithm $A$ if $A(G) \leq OPT(G) + C$ for some constant integer $C$ and any instance, $G$ of the vertex cover.)
	\item The VERTEX-COVER problem and the $NP$-complete CLIQUE problem are complementary in the sense that an optimal vertex cover is the complement of a maximum-size clique in the complement graph. Does this relationship imply that there is a polynomial-time approximation algorithm with a constant approximation ratio for the CLIQUE problem? Justify your answer.
	
	\end{enumerate}
	
		
\end{enumerate}



\end{document}
