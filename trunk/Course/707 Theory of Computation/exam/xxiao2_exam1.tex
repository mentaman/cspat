%% Based on a TeXnicCenter-Template by Gyorgy SZEIDL.
%%%%%%%%%%%%%%%%%%%%%%%%%%%%%%%%%%%%%%%%%%%%%%%%%%%%%%%%%%%%%

%------------------------------------------------------------
%
\documentclass{article}%
%Options -- Point size:  10pt (default), 11pt, 12pt
%        -- Paper size:  letterpaper (default), a4paper, a5paper, b5paper
%                        legalpaper, executivepaper
%        -- Orientation  (portrait is the default)
%                        landscape
%        -- Print size:  oneside (default), twoside
%        -- Quality      final(default), draft
%        -- Title page   notitlepage, titlepage(default)
%        -- Columns      onecolumn(default), twocolumn
%        -- Equation numbering (equation numbers on the right is the default)
%                        leqno
%        -- Displayed equations (centered is the default)
%                        fleqn (equations start at the same distance from the right side)
%        -- Open bibliography style (closed is the default)
%                        openbib
% For instance the command
%           \documentclass[a4paper,12pt,leqno]{article}
% ensures that the paper size is a4, the fonts are typeset at the size 12p
% and the equation numbers are on the left side
%
\usepackage{amsmath}%
\usepackage{amsfonts}%
\usepackage{amssymb}%
\usepackage{graphicx}
%-------------------------------------------
\newtheorem{theorem}{Theorem}
\newtheorem{acknowledgement}[theorem]{Acknowledgement}
\newtheorem{algorithm}[theorem]{Algorithm}
\newtheorem{axiom}[theorem]{Axiom}
\newtheorem{case}[theorem]{Case}
\newtheorem{claim}[theorem]{Claim}
\newtheorem{conclusion}[theorem]{Conclusion}
\newtheorem{condition}[theorem]{Condition}
\newtheorem{conjecture}[theorem]{Conjecture}
\newtheorem{corollary}[theorem]{Corollary}
\newtheorem{criterion}[theorem]{Criterion}
\newtheorem{definition}[theorem]{Definition}
\newtheorem{example}[theorem]{Example}
\newtheorem{exercise}[theorem]{Exercise}
\newtheorem{lemma}[theorem]{Lemma}
\newtheorem{notation}[theorem]{Notation}
\newtheorem{problem}[theorem]{Problem}
\newtheorem{proposition}[theorem]{Proposition}
\newtheorem{remark}[theorem]{Remark}
\newtheorem{solution}[theorem]{Solution}
\newtheorem{summary}[theorem]{Summary}
\newenvironment{proof}[1][Proof]{\textbf{#1.} }{\ \rule{0.5em}{0.5em}}

\begin{document}

\begin{flushleft}
\textbf{Course:} CSC707, Automata, Computability and Computational Theory\\
\textbf{EXAM 1}: Countability, closure properties of countable sets. Complexity theory, NP-completeness, polynomial-time reducibility, self-reduction, approximability.\\
\textbf{Submission:} Home-take\\
%\textbf{Duration:} 75 minutes\\
\textbf{FULL NAME:} Xusheng Xiao \\
\end{flushleft}

\begin{center}
\fbox{\textbf{Due Date: February 25, 11:59 PM}
\textbf{Mid-Term Exam, Spring, 2010}}\\
\end{center}

\noindent{\hrulefill}

\begin{enumerate}
	\item Provide a solution to \textbf{ONE} problem in \textbf{EACH} category; a total of \textbf{FOUR} categories.
	\item Solutions to the other problems in each category will\textbf{ NOT} be graded. 
	\item This is an open-textbook (main course book), open-notes, open-homeworks, BUT \textbf{closed-internet} exam. You may \textbf{NOT} discuss the exam with any one. The exam is an \textbf{INDIVIDUAL} effort.
\end{enumerate}
\noindent{\hrulefill}


\bigskip

\begin{enumerate}

	%---------- PROBLEM 1 --------------
	\item Prove or disprove the countability of each of the following sets (\textbf{25 points}: identifying which class (5 points), proof idea (5 points), complete proof (15 points)):	
	\begin{enumerate}
		\item The set of all regular languages over $\{0, 1\}$.
		\item The set of all languages over the alphabet $\{0, 1\}$
		\item The set of all infinite length strings over the three letter alphabet $\{0, 1, 2\}$
	\end{enumerate}
	
	I would prove that the set of all infinite length strings over the three letter alphabet $\{0, 1, 2\}$ is uncountable.

	\begin{theorem}
	The set $S$ of all infinite length strings over the three letter alphabet $\{0, 1, 2\}$ is uncountable.
	\end{theorem}
	
	\begin{proof}
	Prove by contradiction using diagonalization. Assume that $S$ is countable. By defnition of a countable set, there exists a bijection
function $f : N \rightarrow S$. Therefore, the elements of $S$ can be enumerated as:
$$S=\{s_1,s_2,\ldots,s_n,\ldots\}$$

  Each element of $S$ can be represented as follows:
  $$s_{j}=a_{j1}a_{j2}a_{j3}\ldots a_{jn} \ldots | a_{jn} \in \{0,1,2\},\forall n \in N$$

	The elements of $S$ can be represented as a matrix, where each row $j$ corresponds to a infinite length of string $s_j$ and each element in row $j$ and column $k$ is an alphabet $a_{jk} \in \{0,1,2\}$
	\begin{table}
  \begin{tabular}{l|l||l|l|l|l|l|l}
  \hline
$n \in N$&$s_n$&1&2&3&\ldots&$j$&\ldots \\ \hline
1 $\rightarrow$ & $s_1$ & $a_{11}$ & $a_{12}$& $a_{13}$ & \ldots & $a_{1j}$ & \ldots \\ \hline
2 $\rightarrow$ & $s_2$ & $a_{21}$ & $a_{22}$& $a_{23}$ & \ldots & $a_{2j}$ & \ldots \\ \hline 
3 $\rightarrow$ & $s_3$ & $a_{31}$ & $a_{32}$& $a_{33}$ & \ldots & $a_{3j}$ & \ldots \\ \hline
\ldots & \ldots  & \ldots  & \ldots & \ldots  & \ldots & \ldots  & \ldots \\ \hline
$j \rightarrow$ & $s_j$ & $a_{j1}$ & $a_{j2}$& $a_{j3}$ & \ldots & $a_{jj}$ & \ldots \\ \hline
\ldots & \ldots  & \ldots  & \ldots & \ldots  & \ldots & \ldots  & \ldots \\ \hline

\end{tabular}
\end{table}	
 
 Construct a new element $s_{new}$ from $S$ as follows:
 $$s_{new}=a_{k1}a_{k2}a_{k3}\ldots a_{kn} \ldots | a_{kn} \in \{0,1,2\},a_{kn} \neq a_{nn},\forall n \in N$$
 By definition of $S$, $s_{new} \in S$. It means that $\exists j \in N$ such that $s_{new}=s_j$. But by definition of $s_{new}$, $a_{kj} \neq a_{jj}$. Thus, $s_{new} \neq s_j$ $\Rightarrow$ $s_{new} \notin S$. Contradiction.
 
	\end{proof}
		
	%---------- PROBLEM 2 --------------
	\item \textbf{$NP$-completeness, 25 points}: Verification Step (5 points), Reduction Step (10 points), Correctness Step (10 points): Prove that the problem is $NP$-complete.
	
	\begin{enumerate}
		\item BIGGER-CLIQUE = $\{\langle G_1, G_2\rangle \mid $ the largest clique of graph $G_1$ is larger than every clique of graph $G_2\}$
		\item NP-PATH = {$<G, s, t, k>$ | $G$ is an undirected graph containing a simple path of length at least $k$ from $s$ to $t$.} (Hint: \textit{Use the fact that the Hamiltonian Path problem for undirected graphs is $NP$-complete}.)
		\item HALF-CLIQUE = {$<G>$ | $G$ is an undirected graph having a complete subgraph with �$\left\lfloor n/2 \right\rfloor$ nodes, where $n$ is the number of nodes in $G$}.
		%\item Longest Path (Reduction from Hamiltonian Path)
		%\item MAX-CUT (See hint for Problem 7.25 in Sipser's book)
	\end{enumerate}

  I would prove that NP-PATH is $NP$-complete.
  
  INSTANCE: A graph $G = (V,E)$, a starting vertex $s \in V$, an ending vertex $t \in V$ and a positive integer $k$
  QUESTION: Does $G$ have a simple path $P=\{e_1,e_2,\ldots,e_k | e_n \in E, 1 \leq n \leq k \}$ of length at least $k$ from $s$ to $t$?
  
  \begin{theorem}
  
  NP-PATH is $NP$-complete.
  
  \end{theorem}
  
  \begin{proof}
  
  \textit{Step 1: Verification}
  
  $NP-PATH$ can be verified as follows: 
  
\begin{enumerate}
	\item Count the number of the edges of the given path, this takes $O(|P|)$. \
	\item If $|P|$ is less than $k$, then return ``no''.
	\item Else, create an integer array $C$ of size $|V|$, and assign 0 to each element in the array, which requires $O(|V|)$. In this way, each vertex can be mapped to one element of the array. For each edge in the path $P$, we can know the two endpoints of the edge and increase the value of the corresponding elements in the array $C$, which takes tiem of $O(|P|)$. Since the path starts from $s$ and ends at $t$, the value of the corresponding elements in the array must be 1. All the other vertices in the path must be mapped to the elements whose value is 2. 
	\item All the egdesAfter all the edges are counted, we can check each element in the array $C$, if the element that mapped to $s$ and $t$ are not 1, then return ``no''. If there are $|P| - 1$ elements whose count is 2, then return ``yes''. Otherwise, return ``no''. This takes time $O(|V|)$.
	
	Since all these steps can be completed in polynomial time, the total time required for verifying NP-PATH can be completed in polynomial time. Thus, $NP-PATH$ is in $NP$.
\end{enumerate} 
  

  \textit{Step 2: Reduction}
  
  We now show that $NP-PATH$ is $NP-Hard$. Let $<G, s, t>$ be ANY instance of $Hamiltonian Path$, let $n=V(G)$. We produce SOME instance $<G', s', t', k'>$ of 
$NP-PATH$ as follows:

Construct $G'=G$,$s'=s$,$t'=t$,$k=n-1$, Return $<G, s, t,k>$. This construction is in $O(|V|+|E|)$, which is polynomial.
  
  \textit{Step 3: Correctness}
  
  If there is a $Hamiltonian Path$ from $s$ to $t$, then there is a simple of length $n-1$, which is a $NP-PATH$ of length at least $n-1$ from $s$ to $t$. Similarly, if there is a $NP-PATH$ of length at least $n-1$ from $s$ to $t$, then there is a simple path of length $n-1$ that visits every vertiex in the graph, which means that it is a $Hamiltonian Path$ from $s$ to $t$. 
  
  
  \end{proof}


%---------- PROBLEM 3 --------------

	\item \textbf{Self-reducibility, 25 points}: 
	\begin{enumerate}
		\item Find an isomorphism between graphs $G_1$ and $G_2$ and provide time complexity, or state that none exists.  (An isomorphism is a bijection $\phi : V(G_1) \to V(G_2)$ such that $(v_1, v_2) \in E(G_1)$ iff $(\phi(v_1), \phi(v_2)) \in E(G_2)$.)  Assume that you have a decision algorithm $D(G, G')$ that decides whether $G$ and $G'$ are isomorphic in $O(f(|V(G)| + |V(G')|))$ time.
		
		\item Given a set of integers $A$, find a subset of $A$ that sums to zero and provide time complexity, or state that none exists.  Assume that you have a decision algorithm $D(S)$ that decides whether such a subset exists in $O(f(|S|))$ time.	
	\end{enumerate}

  I would solve the problem (b).
  
\begin{enumerate}
	\item  \textbf{Given}: A set of integers $A$, and decision algorithm $D(S)$ that decides whether a subset that sums to zero exists in $O(f(|S|))$ time.
  \item \textbf{Return}: If $A$ has a subset that sums to zero, return the instance of it,
else return empty.
  \item \textbf{Procedure}: 
  
\begin{enumerate}
  \item Mark every integer in $A$ white.
	\item Call $D(A)$
	\item If return ``no'', then return empty. Otherwise, go to step iv
	\item Construct $A'$ by removing an arbitrary integer $i$ that is marked white.
	\item Call $D(A')$, if return ``yes'', $A=A'$ and repeat step iv until there is no integer marked white.
	\item If return ``no'', mark $i$ black, put $i$ back to $A$ and repeat step iv until there is no integer marked white.
\end{enumerate}
  
  After these steps, all the integers left in $A$ is a subset that sums to zero. We could call $D(A)$ for $|A|$ times, thus the time complexity is $O(|A|*f(|S|))$
\end{enumerate}


%---------- PROBLEM 4 --------------
	
	\item \textbf{Approximability, 25 points}: 
	\begin{enumerate}
	\item  If $P \neq NP$, then Vertex Cover does not allow any absolute approximation. (There is an absolute approximation algorithm $A$ if $A(G) \leq OPT(G) + C$ for some constant integer $C$ and any instance, $G$ of the vertex cover.)
	\item The VERTEX-COVER problem and the $NP$-complete CLIQUE problem are complementary in the sense that an optimal vertex cover is the complement of a maximum-size clique in the complement graph. Does this relationship imply that there is a polynomial-time approximation algorithm with a constant approximation ratio for the CLIQUE problem? Justify your answer.
	
	\end{enumerate}
	
	I would solve problem (a).
	
	 \begin{theorem}
  
  If $P \neq NP$, then Vertex Cover does not allow any absolute approximation. 
  
  \end{theorem}
  
  \begin{proof}
  Prove by contradiction. 
  
  Suppose that given any graph $G$, there is an absolute approximation $A$ such that $A(G) \leq OPT(G) + C$ for some constant $C$. For any graph $G$, we can construct a graph $G'$ by making $C+1$ copies of $G$ and no two copies of $G$ are connected to each other. Then a vertex cover in $G'$ consists of a vertex cover in each copy of $G$ and $OPT(G') = (C+1)OPT(G)$. Call $A(G')$, we can get an approximation
to a vertex cover of $G'$ whose size is at most $OPT(G') + C$. This vertex cover contains a vertex cover for each copy of $G$ in $G'$. By the defintion of $A$, we have:

 $$A(G') \leq OPT(G') + C$$
 $$\Rightarrow A(G') \leq (C+1)OPT(G) + C$$
 $$\Rightarrow (C+1)A(G) - (C+1)OPT(G) \leq C$$ 
 $$\Rightarrow \frac{A(G')}{C+1} - OPT(G) \leq \frac{C}{C+1}$$ 

Since the solution to vertex cover can only be integer, we can have:
$$\frac{A(G')}{C+1} - OPT(G) \leq \frac{C}{C+1} \Rightarrow  A(G) - OPT(G) \leq 0 \Rightarrow A(G) - OPT(G) = 0$$ 

In this way, we can obtain an exact solution to vertex cover in polynomial time. Since vertex cover is a $NP$-complete problem, if it can be solved in polynomial time, $P=NP$, which contradicts with our assumption that $P \neq NP$. 
  
  \end{proof}
		
\end{enumerate}



\end{document}
