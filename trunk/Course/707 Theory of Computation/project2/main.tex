\documentclass[times, 10pt,onecolumn]{article} 
\usepackage{latex8}
\usepackage{times}
\usepackage{url}
\usepackage{graphicx}
%\usepackage{amsfonts, amsthm}

\title{Use Context-Free Grammar to Assist Symbolic Execution in Software Testing}
\author{
Xusheng Xiao\\
\small{xxiao2@ncsu.edu}\\
\and
Xi Ge\\
\small{xge@ncsu.edu}\\
\and
Da Young Lee\\
\small{dlee10@ncsu.edu}
}
\date{March 10, 2010}


\newcommand{\N}{\mathbb{N}}
\newcommand{\Z}{\mathbb{Z}}
\newcommand{\R}{\mathbb{R}}

\begin{document}
\maketitle


\begin{abstract}
Symbolic execution is a way to track programs symbolicly rather than executing them with actual input value. With the impressive progress in constraint solvers, concolic path-based testing tools have literally blossomed up by combining both concrete and symbolic execution, which makes it possible to perform automatic path-based testing on large scale programs.
Many DSE tools could be reduced to the constraint generation and constraint solver phases. Context free grammar plays a critical role in language definition and constraint specification. In this report, we give several examples of the usage of context free grammar in helping dynamic symbolic execution like grammar-based whitebox fuzzing, a kind of solver for String constraints called hampi, and symbolic grammar. We also discuss the significant influence of context free grammar upon these testing aspects and the early researchers who contributed. 
\end{abstract}
\section{Introduction} 
Symbolic execution~\cite{symbolic} is a way to track programs symbolicly rather than executing them with actual input value. Concolic path-based testing tools have literally blossomed up recently \cite{extenjpf,structural,mixed,exe,fuzzing,pex} with the impressive progress in constraint solvers. Concolic path-based testing tools combine both concrete and symbolic execution (referred as concolic execution~\cite{dart,cute} or mixed execution~\cite{mixed}), which makes it possible to perform automatic path-based testing on large scale programs. By executing the program under test with concrete values while performing symbolic execution, symbolic constraints on the inputs can be collected from the predicates in branch statements, forming an expression, called path condition. To explore new paths, part of the constraints in the collected path conditions are negated to obtain new path conditions, which are sent to a constraint solver to compute test inputs for new paths. In theory, all feasible execution paths will be exercised eventually through the iterations of constraint collection and constraint solving in DSE.

Whitebox fuzzing~\cite{fuzzing} executes the program under test with an initial, well-structured input, both concretely and symbolically. Along the execution, symbolic execution collects constraints on program inputs from the predicates in the conditional statements. The conjunction of these constraints of a execution path form an expression, called path condition. Satisfying the negation of each constraint in the path condition defines new inputs that exercise different control paths. Whitebox fuzzing repeats this process for the newly created inputs, with the goal of exercising many different control paths of the program under test and finding defects as fast as possible using various search heuristics. In practice, the search is usually incomplete because the number of feasible control paths grows exponentially with number of conditional statements in the program under test and because the precision of symbolic execution, constraint generation and solving is inherently limited. However, whitebox fuzzing has been shown to be very effective in finding new security vulnerabilities in several applications.

Hampi \cite{hampi} is designed and implemented as a constraint solver for string-manipulating programs. Hampi constraints express membership by regular language, fixed size context-free language. It may contain a fixed size string variable, context-free language definition, regular language definition and operations, and language-membership predicates. Given a set of string constraints over a string variable, Hampi outputs a string that satisfies all the constraints or reports that the constraints are unsatisfiable. Hampi is used as a component in testing, analysis, and verification applcations. Hampi can also be used to solve the intersection, containment, and equivalence problems for regular and fixed size context free languages.

CESE \cite{CESE} is an approach that targets at generating test inputs for programs accepting inputs, whose language can be described using context-free grammars. In particular, CESE is a hybrid approach that combines the advantages of two different approaches: specification-based enumerative test generation \cite{yagg} and dynamic symbolic execution \cite{system,symbolic,Test,counter,random}. CESE proposes symbolic grammar, which are in the form of context-free grammars. Symbolic grammar includes symbolic variables for terminals instead of actual concrete values, which are generally described using regular expressions. CESE automatically generates concrete values for symbolic variables in symbolic grammar by exploring the program under test using dynamic-symbolic-execution-based approaches. The primary advantage of symbolic grammars is that they reduce the state-space of possible values for inputs significantly.


\section{hampi}
A lot of automatic analysis, testing, and verification tools can be reduced to a constraint generation phase and a constraint solving phase. The seperation of these two phases have leveraged more reliable and maintainable tools. In addition to that, increasing availability and efficiency of many off-the-shelf constraint solver makes the approach even more compelling. Hampi \cite{hampi} is designed and implemented as a constraint solver for string-manipulating programs. Hampi constraints express membership in regualar language, fixed size context-free language and membership predicates. Given a set of constraints, hampi will give the string that satisfies all the constraints or report unsatisfiable. The experiment showes that Hampi is efficient in finding SQL injections by static and dynamic ananlysis on web applications and powerful in automated bug finding in system testing of c programs.
 
Many programs, like web applications, take string as inputs, manipulate them and then use them in sensitive operations as database queries. String constraint solver plays a very important role in automatic testing\cite{}, verifying the correctness of program outputs\cite{}, and finding security faults\cite{}. Writing a string constraints solver is a very time-consuming work, and integrating it will cause less maintainable system. Therefore, Hampi is designed and implemented to meet this need as a third-party module that can be easily integrated into a variety of applications.     

Hampi constraints express membership by regular language, fixed size context-free language. It may contain a fixed size string variable, context-free language definition, regular language definition and operations, and language-membership predicates. Given a set of string constraints over a string variable, Hampi outputs a string that satisfies all the constraints or reports that the constraints are unsatisfiable. Hampi is used as a component in testing, analysis, and verification applcations. Hampi can also be used to solve the intersection, containment, and equivalence problems for regular and fixed size context free languages.

A key feature for Hampi is that the fixed-sizing of regular and context free grammar. This feature differentiate Hampi with other string constraints solvers that used in many testing and analysis applications. Fixed-sizing is not a handicap for a constraint solver, but allows more expressive languages and many operations upon context-free language that would be undecidable without fixed-sizing. Fixed-sizing also renders the satisfiability problem solved by Hampi more tractable.

Hampi works in four steps: the first is to normalize the input constraints to formal forms which are called core string constraints. The core string constraints are expressions of the form $v\in R$ or $v\notin R$, where $v$ is the input fixed-size string varible, and $R$ is the regular expression. Second, translate the core string constraints into quantifier-free logic of bit-vectors which are fixed-size, ordered lists of bits. Third, hand over the bit-vector constraints logic that Hampi uses to STP\cite{} which is a constraints solver for bit-vectors and arrays. Fourth, according to the report provided by STP, we get the result whether the original string constraints is satisfiable, if yes, generate a satisfying assignment in its bit-vector language and output a string solution; otherwise, report unsatisfiable. 

We discuss the prominent feature and illustrate its language input by example. Hampi input enables the encoding of string constraint generated from the typical testing and security applications. The language supports the declaration of fixed-size variables and constraints, regular language operations, membership predicates, and the declaration of context free and regular languages, temporaries and constraints.

Var is the string variable declared of the size specified. If all the constraints of the Hampi are satisfiable, var will be afforded value meets all the constraints. Sometimes, the application requires the constraint solver to consider all the string up to a fixed size. This end could be achieved by one of the following two ways: (1) repeatedly applying Hampi for different fixed size up to the given maximum size; (2) adjusting the constraints to allow "padding" of the variable. 

Hampi allows the standard notation Extended Backus-Naur Form(EBNF) to specify context free grammar in input. Terminals are enclosed in double quotes(e.g., "SELECT"), and productions are seperated by vertical bar symble (|). Grammars may contain special symbols for repetition (+ and *) and character ranges(e.g.,[a-z]).

Reg is the declaration of regular language. Regular languages are defined as following four regular expressions:(i) a singleton set with a string constant; (ii) a concatenation or union of regular languages; (iii) a repetition of a regular language; (iv) a fixed sizing of a context free language. Every regular language can be defined by the first three of these operations.  

Vals are temporary variables that act as shortcuts for expressing constraints on expressions that are concatenations of the string varibles and constants.

Assert is the key word that used by Hampi to express the membership of strings in regular languages.

After parsing all the Hampi input, Hampi normalize the string constraints into core form. The core string constraints are an internal intemediate representation that is easier to be encoded into bit-vector logic than raw Hampi input is. A core string constraints specifies membership in a regular language. A core string constraint is expressed in the form $StrExp\in RegExp or StrExp\notin RegExp$, where $StrExp$ is an expression composed of concatenations of string constants and occurrences of the string variable, and $RegExp$ is a regular expression.

The algorithm Hampi uses to create regular expressions that specify the set of strings of fixed length that are derivable from the context free grammar:
\begin{enumerate}
	\item Expand all special symbols in the grammar, like repitition, option, character range.
	\item Remove $\epsilon$ productions.
	\item Take the following steps to construct the regular expression that encodes all fixed size strings of the grammar: (i) precompute the shortest and longest size of the string that can be generated from every nonterminal(i.e. upper bound and lower bound). (ii) given a size $n$ and a nonterminal $N$, examine all the possible productions for $N$. For each $N\rightarrow S_1S_2...S_k$, where each $S_i$ could be nonterminal or terminal, enumerate all possible partitions of $n$ characters to $k$ grammar symbols. Then, create sub-expressions recursively and combine the sub-expressions together with a concatenation operator. Memoization the intermediate results makes this process scalable. 
\end{enumerate}   
The next phase of Hampi is to encode the core string constraints as fomulas in the logic of fixed-size bit-vectors. A bit-vector is fixed size, ordered list of bits. The fragment of bit vector logic that is used by Hampi contains standard boolean operations, extracting sub-vectors, and comparing bit vectors. Hampi asks STP for a satisfying assignment to the resulting bit-vector formula. If STP found one, Hampi decodes it and produce a string solution for the input constraints, otherwise Hampi will terminate and report that the string constraints is not satisfiable. The encode procedure is as follows:
\begin{enumerate}
	\item The constant string values are enforced by Hampi as relevant elements of the bit-vector variable.
	\item Hampi encodes the union operator(+) as a disjunction in the bit-vector logic.
	\item Hampi encodes the concatenation operator by enumerating all possible distributions of the characters to the sub-expressions, encoding the sub-expression recursively, and combining the sub-formulas in a conjunction.
	\item The Kleene Star will be encoded similarly to concatenation.
	\item After STP finds a solution to the bit-vector formula (if exists), Hampi decodes the solution by reading 8-bit sub-vectors as consecutive ASC2 characters.
	 

\end{enumerate}

\section{Grammar-based Whitebox Fuzzing Using Symbolic Execution}
Whitebox fuzzing~\cite{fuzzing} executes the program under test with an initial, well-structured input, both concretely and symbolically. Along the execution, symbolic execution collects constraints on program inputs from the predicates in the conditional statements. The conjunction of these constraints of a execution path form an expression, called path condition. Satisfying the negation of each constraint in the path condition defines new inputs that exercise different control paths. Whitebox fuzzing repeats this process for the newly created inputs, with the goal of exercising many different control paths of the program under test and finding defects as fast as possible using various search heuristics. In practice, the search is usually incomplete because the number of feasible control paths grows exponentially with number of conditional statements in the program under test and because the precision of symbolic execution, constraint generation and solving is inherently limited. However, whitebox fuzzing has been shown to be very effective in finding new security vulnerabilities in several applications.

In practice, the current effectiveness of whitebox fuzzing is limited when testing applications with highly structured
inputs, e.g., compilers and interpreters. These applications process their inputs in stages, such as lexing, parsing and evaluation. Because of the enormous number of control paths in early processing stages, whitebox fuzzing rarely reaches parts of the application beyond these first stages. For instance, there are many possible sequences of blank-spaces/tabs/carriagereturns/etc. separating tokens in most structured languages, each corresponding to a different control path in the lexer. In addition to path explosion, symbolic execution may fail already in the first processing stages. For instance, lexers often detect language keywords by comparing their pre-computed, hard-coded hash values with the hash values of strings read from the input; this effectively prevents symbolic execution and constraint solving from ever generating input strings that match those keywords, since hash functions cannot be inversed (i.e., given a constraint x == hash(y) and a value for x, one cannot compute a value for y that satisfies this constraint).

In \cite{grammar}, Godefroid et al. propose a new approach, called \textit{grammar-based whitebox fuzzing}, which enhances whitebox fuzzing with a grammar-based specification of valid inputs. They present a dynamic test generation algorithm where symbolic execution directly generates grammar-based constraints whose satisfiability is checked using a custom grammar-based constraint solver. Their algorithm consists of two key components:

\begin{enumerate}
\item Generation of higher-level symbolic constraints, expressed in terms of symbolic grammar tokens returned by the
lexer, instead of the traditional~\cite{dart,exe,fuzzing} symbolic bytes read as input.
	
\item A custom constraint solver that solves constraints on symbolic grammar tokens. The solver looks for solutions
that satisfy the constraints and are accepted by a given (context-free) grammar.
\end{enumerate}




\section{CESE: Directed Test Generation using Symbolic Grammars}

Majuumar and Xu~\cite{CESE} presented an approach to generate
tests using symbolic grammars. This section introduces
an example of concrete grammar (parsing an arithmetic
operations) and shows differences of test generation
based on random testing,  constrained exhaustive enumeration,
symbolic or concolic execution, and CESE.
 
 \begin{enumerate}  
  \item{Example} \\
They use a calculation example, called \textit{SimpleCalc} to show the effectiveness
of their approach:
\[Expressions~e~::=~(e)~|~e~*~e~|~e~/~e~|~e~\%~e |~e~+~e~|~e~-~e~|~e~\vee~e~|~e~\wedge~e~|~-~e~|~l~|~n\]
\[Letters~l~::= [a~-~zA~-~Z]\]
\[Numbers~n~::= [0~-~9]\]
\textit{SimpleCalc} creates various forms of arithmetic expressions
including letters, operations, and numbers. However, \textit{SimpleCalc} includes
a fault to introduce an error; \textit{SimpleCalc} does
not check a division by zero. For example, when a certain
number is divided by zero (e.g., 7/0), such an operation
causes an error. In order to detect the fault in \textit{SimpleCalc},
testers require an arithmetic expression that is a number divided by zero.  

  \item{Random Testing}\\

Random testing is to generate random test inputs. In a situation where possible input space is large, random testing is not effective to generate a set of test inputs covering particular branches. Consider that our goal is to generate valid inputs using random testing. If an input size of four for \textit{SimpleCalc}, a total number of valid inputs is only 108,066 over 4.2 million possible inputs including that an input of size one is around 4.18 million possible inputs. Therefore, to generate valid inputs randomly is around 0.1\% probability over possible input space. With such a low probability, it is not trivial to generate a test (of interest) to detect the fault in \textit{SimpleCalc}. In addition, \textit{SimpleCalc} has an error when a number is divided by zero (e.g., an expressing including ``/0''). In order to detect the fault, only 372 inputs (that can detect the fault) should be generated over 4.2 million input space. Random testing has very low chance to generate such inputs to detect the fault.
  
  \item{Constrained Exhaustive Enumeration} \\
  
Constrained Exhaustive Enumeration takes an input as specification (e.g., grammar) and generate all valid inputs satisfying an input grammar. In general, such a technique generates all possible valid inputs for given an input grammar. For \textit{SimpleCalc}, valid input space is over 187 million for an input of size six.  Most of generated tests explore the same path of the given grammar; 0 + 0, 0 + 1, ..., 9 + 9 is generated based on the same path instead of different paths. If we want to generate test inputs for exploring different paths, this technique is not effective. 

  
  \item{Symbolic or Concolic Execution} \\

Symbolic execution~\cite{system} and Concolic execution~\cite{dart, cute} (symbolic execution based on concrete variables) explores paths based on solving constraints collected during execution of a program. These techniques have two phases; (1) Tests are generated randomly, and (2) Tests are refined iteratively using constraints collected during execution of a program. However, these techniques are to find all feasible paths satisfying assignments based on constraints. For a size of four, an existing Concolic technique generates a total of 248,523 inputs, and takes more than 30 minutes. Testing of 248,523 inputs to detect the fault in \textit{SimpleCalc} has very low chance.
   
 
  \item{CESE} \\
  
To overcome limitations of previous approaches (e.g., generating a large number of tests to detect a fault), CESE combines the selectiveness of specification-guided test generation and directiveness of symbolic or concolic test generation. CESE is to combine to generate input by symbolic grammar with to execute concolic test. They first introduce a symbolic grammar to covert concrete grammar using symbolic values instead of individual character. For example, in SimpleCalc, they replace Letters $l$ and Numbers $n$ from $[a-zA-Z]$ and $[0-9]$ to $[\alpha][\beta]$ that are symbolic values. Such a conversion reduces the number of strings significantly. For example, previous concrete grammar generates 100 different strings $0/0, 0/1,..., 9/9$ to represent division expressions. However, such a conversion can represent one symbolic string $\beta1 / \beta2$ form to represent all possible division expressions. Given a converted symbolic grammar, they apply concolic execution to find all possible paths. Based on the idea, they replace given a concrete grammar to a symbolic grammar by replacing certain lexical. In addition, they execute concolically for a given grammar; for instance $\alpha1 + \alpha2$ is executed to represent +  operation over $\alpha1$ and $\alpha2$ that exercise different paths of symbolic inputs. As a results, concolic execution exercise 188 paths. In the of performance, on the symbolic grammar, CESE can generate 6,611 and takes less time (that other previous approaches) to compute concolic executions.

 
 
 \end{enumerate}
 

\section{Major Contributors}
The concept of symbolic execution was introduced academically with descriptions of the Select system, proposed by Boyer et al~\cite{select}. In 1976, test data generation using symbolic execution was first proposed by James C. King~\cite{symbolic}. Around the same time, Clarke also did the work on this~\cite{test76}. Their pioneer works open the ways to automatic test data generation by using symbolic execution to do static analysis of code for path safety and prove theorems about code. However, these static ways faced the exponential state space explosion problem, which made it only practical for small programs. In recent years, Koushik Sen, whose paper on concolic testing~\cite{dart} won the ACM SIGSOFT Distinguished Paper Award at ESEC/FSE '05, proposed CUTE and DART tools, blossoming up the path-based automatic test data generation using symbolic execution. With the impressive progress of constraint solvers and concolic path-based testing~\cite{extenjpf,structural,mixed,exe,fuzz,pex}, it is possible to perform automatic path-based testing on large scale programs. Pex~\cite{pex}, a symbolic execution test generation tool for .NET proposed by Nikolai Tillmann, has been used to test .NET core libraries and found serious bugs. To alleviate the classic path explosion problem, many new techniques are also been proposed by Xie~\cite{fitness}, Godefroid~\cite{compositional} and so on.


\bibliographystyle{plain}
\bibliography{references}
\end{document}