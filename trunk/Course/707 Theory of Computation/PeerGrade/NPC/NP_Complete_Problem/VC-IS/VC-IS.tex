%% Based on a TeXnicCenter-Template by Gyorgy SZEIDL.
%%%%%%%%%%%%%%%%%%%%%%%%%%%%%%%%%%%%%%%%%%%%%%%%%%%%%%%%%%%%%

%------------------------------------------------------------
%
\documentclass{article}%
%Options -- Point size:  10pt (default), 11pt, 12pt
%        -- Paper size:  letterpaper (default), a4paper, a5paper, b5paper
%                        legalpaper, executivepaper
%        -- Orientation  (portrait is the default)
%                        landscape
%        -- Print size:  oneside (default), twoside
%        -- Quality      final(default), draft
%        -- Title page   notitlepage, titlepage(default)
%        -- Columns      onecolumn(default), twocolumn
%        -- Equation numbering (equation numbers on the right is the default)
%                        leqno
%        -- Displayed equations (centered is the default)
%                        fleqn (equations start at the same distance from the right side)
%        -- Open bibliography style (closed is the default)
%                        openbib
% For instance the command
%           \documentclass[a4paper,12pt,leqno]{article}
% ensures that the paper size is a4, the fonts are typeset at the size 12p
% and the equation numbers are on the left side
%
\usepackage{amsmath}%
\usepackage{amsfonts}%
\usepackage{amssymb}%
\usepackage{graphicx}
%-------------------------------------------
\newtheorem{theorem}{Theorem}
\newtheorem{acknowledgement}[theorem]{Acknowledgement}
\newtheorem{algorithm}[theorem]{Algorithm}
\newtheorem{axiom}[theorem]{Axiom}
\newtheorem{case}[theorem]{Case}
\newtheorem{claim}[theorem]{Claim}
\newtheorem{conclusion}[theorem]{Conclusion}
\newtheorem{condition}[theorem]{Condition}
\newtheorem{conjecture}[theorem]{Conjecture}
\newtheorem{corollary}[theorem]{Corollary}
\newtheorem{criterion}[theorem]{Criterion}
\newtheorem{definition}[theorem]{Definition}
\newtheorem{example}[theorem]{Example}
\newtheorem{exercise}[theorem]{Exercise}
\newtheorem{lemma}[theorem]{Lemma}
\newtheorem{notation}[theorem]{Notation}
\newtheorem{problem}[theorem]{Problem}
\newtheorem{proposition}[theorem]{Proposition}
\newtheorem{remark}[theorem]{Remark}
\newtheorem{solution}[theorem]{Solution}
\newtheorem{summary}[theorem]{Summary}
\newenvironment{proof}[1][Proof]{\textbf{#1.} }{\ \rule{0.5em}{0.5em}}

\begin{document}

\begin{flushleft}
\textbf{Course:} CSC707, Automata, Computability and Computational Theory\\
\textbf{Reduction Homework}: NP-complete problems \\
\textbf{Submission:} Use Wolfware\\
\textbf{File Format:} LaTeX and PDF\\
\textbf{Students:} Xusheng Xiao xxiao2@ncsu.edu, Xi Ge xge@ncsu.edu, Ling Chen lchen10@ncsu.edu \\
\end{flushleft}

\begin{center}
\fbox{\textbf{Due Date:} \textbf{2:00 A.M. (EST), Tuesday, February 9, 2010}}\\
\begin{enumerate}
	\item Provide any feedback/questions you may have on this homework (\textbf{optional}).
	\item Using LaTeX is required.
\end{enumerate}\end{center}

\noindent{\hrulefill}

\bigskip

\begin{enumerate}

  \item Given a NP-complete problem, Vertex Cover, show that the Independent Set is $NP$-complete.\\
  Independent Set is defined as follows:\\
  INSTANCE: A graph $G=(V,E)$ and a positive integer $k \leq |V|$.\\
  QUESTION: Is there a subset S of k vertices in G such that no pair of vertices in S is connected by an edge in G?
  
\textbf{Solution:}
\begin{enumerate}
	\item (Verification):Show that Independent Set is in NP.
	\\ Given an independent set $C\subseteq V, |C|=n$ for a graph $G=(V,E)$, we can verify it using the following pseudo-code: \\
	
		$\forall u \in C$, \\
	\hspace*{0.2in}$\forall v \in C$, \\
	\hspace*{0.4in}check whether $<u,v> \in E$ \\

	Checking whether $<u,v> \in E$ can be done in $T(|E|)$. So the verification algorithm takes $O(n^{2}|E|)$.
	
%	
%	The verification algorithm guesses $C\subseteq V$ and check that whether C is an independent set of G=(V,E). If the test succeeds then the algorithm accepts, otherwise it rejects. To check whether C is an independent set of G=(V,E) only needs one travesal on G which takes $O(V+E)$, so the verification algorithm takes $O(n\cdot (V+E))$.\\
	
	\item (Reduction):Show that Independent Set is NP-hard.
	\\Given a G has a VC of size k, we should construct a graph G' has Independent Set of size k'.
	\\Construction Process: Given VertexCover(G,k) where $V_{1}$ is the vertex cover and $k=|V_{1}|$, we set $G'=G$ and $k'=|V|-k$, then we could return the answer to IndependentSet(G',k') where $V-V_{1}$ is the independent set. This takes constant time.\\
	
	\item (Correctness):Show that Independent Set is NP-hard.
	\\We need to show that $G$ has a vertex cover of size $k$ if and only if it has an Independent Set of size $k'=|V|-k$.\\
	\\Assume $G(V,E)$ has a vertex cover $C$ of size $k$. Consider two vertices $u\in V-C$ and $v\in V-C$, we can know that $e=<u,v>\notin E$ since $C$ is a vertex cover of $G$. Therefore, no two vertices in $V-C$ are connected by an edge. So $V-C$ is an independent set with size $k'=|V|-k$.\\	
		
	Assume $G$ has an Independent Set $S$ of size $k'=|V|-k$.
	\\$\forall e\in E, e=<u,v>$, $S$ is independent set$\Rightarrow$ $u\notin S$ or $v\notin S$ $\Rightarrow$ $u\in V-S$ or $v\in V-S$ $\Rightarrow$ $V-S$ covers $e=<u,v>$.\\
	
	
\end{enumerate}
\end{enumerate}

\end{document}
