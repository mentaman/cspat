\documentclass[times, 10pt,onecolumn]{article} 
\usepackage{latex8}
\usepackage{times}
\usepackage{url}
\usepackage{graphicx}
%\usepackage{amsfonts, amsthm}

\title{Use Context-Free Grammar to Assist Symbolic Execution in Software Testing}
\author{
Xusheng Xiao\\
\small{xxiao2@ncsu.edu}\\
\and
Xi Ge\\
\small{xge@ncsu.edu}\\
\and
Da Young Lee\\
\small{dlee10@ncsu.edu}
}
\date{March 10, 2010}


\newcommand{\N}{\mathbb{N}}
\newcommand{\Z}{\mathbb{Z}}
\newcommand{\R}{\mathbb{R}}

\begin{document}
\maketitle


\begin{abstract}
Symbolic execution is a way to track programs symbolicly rather than executing them with actual input value. With the impressive progress in constraint solvers, concolic path-based testing tools have literally blossomed up by combining both concrete and symbolic execution, which makes it possible to perform automatic path-based testing on large scale programs. However, these technologies are still suffering from the path explosion problem, since achieving 100\% path coverage is to enumerate all paths between two nodes in a graph, which is well known as a NP-hard problem. To alleviate this path explosion problem and to reduce the computation complexity, different techniques has been proposed, such as pruning search space of symbolic execution, selective symbolic execution and fitness-guided path exploration. In this report, we provide the details on the problem of test data generation and present three techniques to alleviate this path explosion problem. We also discuss the early researchers who contributed to the field of research of test data generation using symbolic execution. \end{abstract}
\section{Introduction} 
Software Engineering is a knowledge-intensive activity, presumably requiring intelligence. Many software engineering activities, such as testing, analysis and debugging, require intensive human intelligence and are error-prone. To reduce human efforts in the activities of software engineering, Artificial Intelligence (AI) techniques, which aims to create computer systems that exhibit some form of human intelligence, are employed to assist or automate various activities of software engineering, such as testing, program analysis, debugging and even self-repair. In this report, we present the details of the application of AI in software engineering and provide a detailed survey of the existing tools and techniques associating with AI in three important software engineering activities: testing,
fault detection and software repair.

Over the past decades, many AI techniques are applied to assist automated software testing, such as constraint solving~\cite{constraintsolving} used in Dynamic Symbolic Execution~\cite{symbolic, dart, cute} for test-input generation, heuristics used to prune search space of test-generation tools~\cite{prune,fitness}, and machine learning used in statistical software testing~\cite{mlinstatistics} and coverage prediction of testing tools~\cite{predictCoverage}. In this report, we provide the details on test generation using symbolic execution and pruning search space of symbolic execution using heuristics.

As the size of complexity of software has grown quickly in past decades, the difficulty of finding and fixng bugs has increased. Recent research works~\cite{wrongDefinition,online} that use AI techniques have advanced the research in reducing the human efforts on fault detection: Shi et al.~\cite{wrongDefinition} proposes an approach to first learn the Definition-Use Invariants and then use the learned knowledge of Definition-Use for detecting concurrency and sequential bugs; Baah et al~\cite{online} and proposes a new machine-learning technique that performs fault detection for deployed software. I plan to study in details how these research works adopt the concepts and techniques of AI to assist the task of fault detection.

By adapting the well-known AI techniques, even the most challenging tasks, debugging and self-repair, can be half or even full automated. Genetic programming, an evolutionary algorithm-based methodology in AI, is used and adapted by Weimer et al.~\cite{geneticPatch}' approach to automatically find patches for programs and automatically fix bugs~\cite{repair}. Inspired by these works, Schulte et al. further propose an approach to study the evolution of assembly code~\cite{evolutionaryComputation} for automated program repair. I plan to investigate these techniques to study how AI can achieve automated software repair.






\section{HAMPI: A Solver for String Constraints}
A lot of automatic analysis, testing, and verification tools can be reduced to a constraint generation phase and a constraint solving phase. The seperation of these two phases have leveraged more reliable and maintainable tools. In addition to that, increasing availability and efficiency of many off-the-shelf constraint solver makes the approach even more compelling. Hampi \cite{hampi} is designed and implemented as a constraint solver for string-manipulating programs. Hampi constraints express membership in regualar language, fixed size context-free language and membership predicates. Given a set of constraints, hampi will give the string that satisfies all the constraints or report unsatisfiable. The experiment showes that Hampi is efficient in finding SQL injections by static and dynamic ananlysis on web applications and powerful in automated bug finding in system testing of c programs.
 
Many programs, like web applications, take string as inputs, manipulate them and then use them in sensitive operations as database queries. String constraint solver plays a very important role in automatic testing\cite{}, verifying the correctness of program outputs\cite{}, and finding security faults\cite{}. Writing a string constraints solver is a very time-consuming work, and integrating it will cause less maintainable system. Therefore, Hampi is designed and implemented to meet this need as a third-party module that can be easily integrated into a variety of applications.     

Hampi constraints express membership by regular language, fixed size context-free language. It may contain a fixed size string variable, context-free language definition, regular language definition and operations, and language-membership predicates. Given a set of string constraints over a string variable, Hampi outputs a string that satisfies all the constraints or reports that the constraints are unsatisfiable. Hampi is used as a component in testing, analysis, and verification applcations. Hampi can also be used to solve the intersection, containment, and equivalence problems for regular and fixed size context free languages.

A key feature for Hampi is that the fixed-sizing of regular and context free grammar. This feature differentiate Hampi with other string constraints solvers that used in many testing and analysis applications. Fixed-sizing is not a handicap for a constraint solver, but allows more expressive languages and many operations upon context-free language that would be undecidable without fixed-sizing. Fixed-sizing also renders the satisfiability problem solved by Hampi more tractable.

Hampi works in four steps: the first is to normalize the input constraints to formal forms which are called core string constraints. The core string constraints are expressions of the form $v\in R$ or $v\notin R$, where $v$ is the input fixed-size string varible, and $R$ is the regular expression. Second, translate the core string constraints into quantifier-free logic of bit-vectors which are fixed-size, ordered lists of bits. Third, hand over the bit-vector constraints logic that Hampi uses to STP\cite{} which is a constraints solver for bit-vectors and arrays. Fourth, according to the report provided by STP, we get the result whether the original string constraints is satisfiable, if yes, generate a satisfying assignment in its bit-vector language and output a string solution; otherwise, report unsatisfiable. 

We discuss the prominent feature and illustrate its language input by example. Hampi input enables the encoding of string constraint generated from the typical testing and security applications. The language supports the declaration of fixed-size variables and constraints, regular language operations, membership predicates, and the declaration of context free and regular languages, temporaries and constraints.

Var is the string variable declared of the size specified. If all the constraints of the Hampi are satisfiable, var will be afforded value meets all the constraints. Sometimes, the application requires the constraint solver to consider all the string up to a fixed size. This end could be achieved by one of the following two ways: (1) repeatedly applying Hampi for different fixed size up to the given maximum size; (2) adjusting the constraints to allow "padding" of the variable. 

Hampi allows the standard notation Extended Backus-Naur Form(EBNF) to specify context free grammar in input. Terminals are enclosed in double quotes(e.g., "SELECT"), and productions are seperated by vertical bar symble (|). Grammars may contain special symbols for repetition (+ and *) and character ranges(e.g.,[a-z]).

Reg is the declaration of regular language. Regular languages are defined as following four regular expressions:(i) a singleton set with a string constant; (ii) a concatenation or union of regular languages; (iii) a repetition of a regular language; (iv) a fixed sizing of a context free language. Every regular language can be defined by the first three of these operations.  

Vals are temporary variables that act as shortcuts for expressing constraints on expressions that are concatenations of the string varibles and constants.

Assert is the key word that used by Hampi to express the membership of strings in regular languages.

After parsing all the Hampi input, Hampi normalize the string constraints into core form. The core string constraints are an internal intemediate representation that is easier to be encoded into bit-vector logic than raw Hampi input is. A core string constraints specifies membership in a regular language. A core string constraint is expressed in the form $StrExp\in RegExp or StrExp\notin RegExp$, where $StrExp$ is an expression composed of concatenations of string constants and occurrences of the string variable, and $RegExp$ is a regular expression.

The algorithm Hampi uses to create regular expressions that specify the set of strings of fixed length that are derivable from the context free grammar:
\begin{enumerate}
	\item Expand all special symbols in the grammar, like repitition, option, character range.
	\item Remove $\epsilon$ productions.
	\item Take the following steps to construct the regular expression that encodes all fixed size strings of the grammar: (i) precompute the shortest and longest size of the string that can be generated from every nonterminal(i.e. upper bound and lower bound). (ii) given a size $n$ and a nonterminal $N$, examine all the possible productions for $N$. For each $N\rightarrow S_1S_2...S_k$, where each $S_i$ could be nonterminal or terminal, enumerate all possible partitions of $n$ characters to $k$ grammar symbols. Then, create sub-expressions recursively and combine the sub-expressions together with a concatenation operator. Memoization the intermediate results makes this process scalable. 
\end{enumerate}   
The next phase of Hampi is to encode the core string constraints as fomulas in the logic of fixed-size bit-vectors. A bit-vector is fixed size, ordered list of bits. The fragment of bit vector logic that is used by Hampi contains standard boolean operations, extracting sub-vectors, and comparing bit vectors. Hampi asks STP for a satisfying assignment to the resulting bit-vector formula. If STP found one, Hampi decodes it and produce a string solution for the input constraints, otherwise Hampi will terminate and report that the string constraints is not satisfiable. The encode procedure is as follows:
\begin{enumerate}
	\item The constant string values are enforced by Hampi as relevant elements of the bit-vector variable.
	\item Hampi encodes the union operator(+) as a disjunction in the bit-vector logic.
	\item Hampi encodes the concatenation operator by enumerating all possible distributions of the characters to the sub-expressions, encoding the sub-expression recursively, and combining the sub-formulas in a conjunction.
	\item The Kleene Star will be encoded similarly to concatenation.
	\item After STP finds a solution to the bit-vector formula (if exists), Hampi decodes the solution by reading 8-bit sub-vectors as consecutive ASC2 characters.
\end{enumerate}
We will further illustrate the procedure by the following example:\\

\indent Var v:2;\\
\indent cfg $E := "()"|EE|"("E")"$;\\
\indent reg $Efixed := fixsize(E, 6)$;\\
\indent Val $q := concat("((",v,"))")$;\\
\indent assert q in Efixed;\\
\indent assert q contains $"())"$;\\

\noindent Step 1: Normalize the constraints to core form. The results after this step are:\\ 
$c_1: ((v))\in ()[()()+(())]+[()()+(())]()+([()()+(())])$\\
$c_2: ((v))\in [(+)]*())[(+)]*$\\
\noindent Step 2: Encode the core form in bit-vector logic. We will illustrate how Hampi would encode constraint $c_1$. First of all, Hampi will create a bit-vector variable $bv$ of size $48=6*8$ bits to represent the lefthand side of $c_1$. Second, the characters are translated into the ASC2 codes corresponding to them, "(" is $40$, and ")" is $41$. Then Hampi encode the lefthand side of $c_1$ as formula $L_1$, by specifying the constant value: $L_1$: $(bv[0]=40)\wedge (bv[1]=40)\wedge (bv[4]=41) \wedge (bv[5]=41)$. Byte $bv[2]$ and $bv[3]$ will be reserved for $v$, a 2-byte variable. Similarly, the right hand side of $c_1$ will be encoded as $D_{1a}\vee D_{1b}\vee D_{1c}$. The entire bit-vector logic of constraint $c_1$ after encoding would be $L_1\vee(D_{1a}\wedge D{1b}\wedge D{1c})$. The final formula that Hampi sends to STP solver is $(C_1\wedge C_2)$.\\
\noindent Step 3: STP finds a solution that satisfies the formula: $bv[0]=40, bv[1]=40, bv[2]=41, bv[3]=40, bv[4]=41, bv[5]=41$. In the decoded ASC2, the solution is $"(()())"$(quote mark is not part of the solution string).
\noindent Step 4: Hampi reads the assignment for variable $v$ off of the STP solution, by decoding the elements of $dv$ that corresponds to $v$, i.e., element 2 and 3. It reports the solutions for $v$ as $")("$.( even though there may be other solutions possible, STP can only find one.)

The performance of Hampi is evaluated through automatic finding of SQL injection attack strings by running a dynamic analysis tool on PHP web applications. Results show that Hampi has successfully replaced Ardilla\cite{}'s custom attack generator, it solves the associated constraints quickly, finds all of the solution of $N\leq 6$, and solved all of the constraints in less than 10 seconds per constraint. 

\section{fuzzing}


\section{symbolicgrammar}


\section{Major Contributors}
The concept of symbolic execution was introduced academically with descriptions of the Select system, proposed by Boyer et al~\cite{select}. In 1976, test data generation using symbolic execution was first proposed by James C. King~\cite{symbolic}. Around the same time, Clarke also did the work on this~\cite{system}. Their pioneer works open the ways to automatic test data generation by using symbolic execution to do static analysis of code for path safety and prove theorems about code. However, these static ways faced the exponential state space explosion problem, which made it only practical for small programs. In recent years, Koushik Sen, whose paper on concolic testing~\cite{dart} won the ACM SIGSOFT Distinguished Paper Award at ESEC/FSE '05, proposed CUTE and DART tools, blossoming up the path-based automatic test data generation using symbolic execution. With the impressive progress of constraint solvers and concolic path-based testing~\cite{extenjpf,structural,mixed,exe,fuzzing,pex}, it is possible to perform automatic path-based testing on large scale programs. Pex~\cite{pex}, a symbolic execution test generation tool for .NET proposed by Nikolai Tillmann, has been used to test .NET core libraries and found serious bugs. To alleviate the classic path explosion problem, many new techniques are also been proposed by Xie~\cite{fitness}, Godefroid~\cite{compositional} and so on.


\bibliographystyle{plain}
\bibliography{references}
\end{document}