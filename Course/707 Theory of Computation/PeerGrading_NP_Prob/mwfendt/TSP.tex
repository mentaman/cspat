\documentclass{article}
\usepackage[pdftex]{graphicx}
\usepackage{geometry, url, color}
\geometry{letterpaper, vmargin={0.80in,1.0in}, hmargin={1.0in}}
\usepackage{verbatim}
\usepackage{alltt}
\usepackage{float}
\usepackage{multirow}
\usepackage{amssymb}
\usepackage{amsmath}
\usepackage{color, ulem}
\definecolor{dkgreen}{rgb}{0, 0.75, 0}
\definecolor{dkred}{rgb}{0.5, 0, 0}
\definecolor{dkpurp}{rgb}{0.25, 0, 0.5}
\newcommand{\add}[1]{\textcolor{dkgreen}{#1}}
\newcommand{\rmv}[1]{\textcolor{red}{\sout{#1}}}
\newcommand{\moveto}[1]{\textcolor{blue}{#1}}
\newcommand{\movefrom}[1]{\textcolor{blue}{\sout{#1}}}
\newcommand{\highlighttext}[1]{\colorbox{yellow}{#1}}
\newenvironment{added}{\color{dkgreen}}{\color{black}}
\newenvironment{removed}{\color{red}}{\color{black}}
\newenvironment{edited}{\color{blue}}{\color{black}}

% Make the contents smaller than normal
% \usepackage[small,compact]{titlesec}
% \usepackage{times}

\setlength{\columnsep}{0.5in}
\pagestyle{empty}
\setlength{\parindent}{1pc}
%Stuff out of art10.sty and modified to conform to IEEE format
\makeatletter

\newcommand{\qed}{\hfill \ensuremath{\Box}}

%as Latex considers descenders in its calculation of interline spacing,
%to get 12 point spacing for normalsize text, must set it to 10 points
\def\@normalsize{\@setsize\normalsize{10pt}\xpt\@xpt
\abovedisplayskip 10pt plus2pt minus5pt\belowdisplayskip
\abovedisplayskip \abovedisplayshortskip \z@
plus3pt\belowdisplayshortskip 6pt plus3pt
minus3pt\let\@listi\@listI}

\def\section{\@startsection {section}{1}{\z@}{1.0ex plus 1ex}{.2ex plus .2ex}{\center\bf}}

%make subsection titles bold and 11 point, 1 blank line before, 1 after
\def\subsection{\@startsection {subsection}{2}{\z@}{1.0ex plus 1ex} {.2ex plus .2ex}{\it}}

\makeatother
\newfloat{codefragment}{p}{cdf}
\floatname{codefragment}{\textbf{\textsf{Code Fragment}}}

\begin{document}

% Uncomment if you don't want date printed
\date{}

%make title bold and 14 pt font (Latex default is non-bold, 16pt)
\title{\Large\bf CSC-707: NP-Complete Proof}

%for single author (just remove % characters)
\author{Jeffery L. Painter, Peter Savitsky, Saurabh Deshpande\\
NCSU, Raleigh, NC}

\maketitle
\thispagestyle{empty}

\add{Graded by Matthew Fendt, Geoffrey Rogers, and Minh Tran, Total of -12 points}

  Prove that the TSP (Traveling Salesman) problem is NP-Complete 
  by reduction to the Hamiltonian Cycle problem. \add{Always reduce from KNOWN to NEW; Reduce from HAMCYCLE to TSP, -2}\\

  Hamiltonian Cycle Problem: \\
  \indent {\bf Input:} An un-weighted graph G \\
  \indent {\bf Output:} Does there exist a {\em{\bf simple}} tour 
                       that visits every vertex of G without repetition?\add{ Repetition of what?  Edges?  Vertices? -1} \\
 
  Prove that TSP is NP-Complete via reduction \\

  {\bf Step 1:} (Verification) Show that $TSP \in NP$ \\

      \indent The TSP problem: 
      $TSP = \{ <G, c, k> : G = (V, E) \text{ is a complete graph }, c \text{ is a function } V \times V \rightarrow Z, $ \\
      \indent $k \in Z \text{ and } G \text{ has a traveling salesman tour with cost at most } k \}$

    Given a tour $h = (v_1, v_2, v_3, \cdots, v_n )$ and a cost $k$, we need to check for two conditions
    \begin{enumerate}
      \item The tour visits every city (vertex) in the graph and returns to the starting point
      \item Cost $k$ is not exceeded by the tour
      \item \add{also need to show no repition of vertices in h, -2}
    \end{enumerate}

    Pseudo-code: \\
    \begin{verbatim}
    Part 1:  ---   O(n^2)
    // identify that every city is visited
    for j = 1 ... |V|:
      vertex_present = false
      for i = 1 ... n
         // test if present
         if ( h[i] contains vertex(j) ):
             vertex_present = true
      if vertex_present == false:
         fails to be a tour

    Part 2:  ---   O(1)
    // test that we end where we began
    if ( h[0] = h[n] ):
      valid_tour = true

    Part 3:  ---   O(n)
    // determine total cost for tour h
    cost = 0
    for i = 0 ... (h-1):
      cost = cost + d( v_i, v_(i+1) )
    if ( cost > k ):
      fails to be a tour of cost <= k
    \end{verbatim}      
    
     Therefore given a graph, we can decide if a tour has cost at most
     $k$ in $T(n) = O(n^2)$. \add{what is n?, -1}Therefore, TSP is verifiable in polynomial time. \\

  {\bf Step 2:} Reduction

  We will use the HAM-CYC problem which is known to be NP-Complete
  and show there exists a polynomial reduction to the TSP problem. \\
  
  Let $G = (V,E)$ be an input instance of the HAM-CYC problem with
  $|V|=n$. \add{Cannot do reduction from specific instance of HAM-CYC.  Should be general instance of HAMCYCLE -5}Then we will construct a complete weighted graph $G' = (V, E')$ such
  that if an edge $e_i \in E$ then the weight of $e_i = 1$ in $E'$
  and if the edge was not in $E$ then the weight of $e_i = 2$. \\
  
  The reduction to TSP can be implemented in polynomial time
  $O(|V| \times |V|) = O(n^2)$ \\
  
  Pseudocode:
  
  \begin{verbatim}
  for i = 1 to |V| do
     for j = 1 to |V| do
        if ( i,j ) in E then w(i,j) = 1 else w(i,j) = 2
  \end{verbatim}
  
  Therefore, $G = (V, E) \rightarrow T(n) = O(n^2) \rightarrow G' = (V, E')$

  

  {\bf Step 3:} Correctness

  {\bf Claim: } The graph $G$ has a Hamiltonian cycle if and only
  if there is a TSP tour of $G'$ of weight exactly $n$. \\
  
  Suppose that the graph $G'$ has a Hamiltonian cycle $h$. Then
  each edge in $h$ belongs to $E$ and has a cost associated with
  it of 1 in $G'$. Also, by definition, every vertex in $V$
  is visited by the cycle $h$. Thus, $h$ is a tour in $G'$ of 
  cost $|V|=n$. \\
  
  Now suppose that the graph $G'$ has a tour $h'$ of cost exactly $n$.
  Then, since the edges in $h'$ \add{you defined h to be a sequence of vertices, not edges, -1}are found in $E'$ and the total cost
  is exactly $n$, then each edge in $h'$ has cost $1$. This
  means that every edge in $h'$ is also in $E$. And since the tour
  implies that every city is visited, we conclude
  that $h'$ must also be a Hamiltonian cycle in graph $G$. \\
  
  \qed
    
\end{document}
