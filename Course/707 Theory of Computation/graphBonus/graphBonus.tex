\documentclass[article, 10pt,onecolumn]{article} 
\usepackage{latex8}
\usepackage{times}
\usepackage{url}
%\usepackage{amsfonts, amsthm}

\title{Graph Bonus}
\author{
Xusheng Xiao\\
\small{xxiao2@ncsu.edu}\\
\and
Xi Ge\\
\small{xge@ncsu.edu}\\
\and
Da Young Lee\\
\small{sthumma@ncsu.edu}
}
\date{January 27, 2010}


\newcommand{\N}{\mathbb{N}}
\newcommand{\Z}{\mathbb{Z}}
\newcommand{\R}{\mathbb{R}}

\begin{document}
\maketitle

\begin{flushleft}
\textbf{Longest Path:}\end{flushleft} Longest path is used to identify critical path in Critical Path Method~\cite{cpm}, a project management method.  is to address the challenge of shutting down chemical plants for maintainance and then restarting the plants once the maintainance had bee completed. 

The steps in CPM project planning includes: 
\begin{enumerate}
	\item Specify the individual activities.
	\item Determine the sequence of those activities.
	\item Draw a network diagram.
	\item Estimate the completion time for each activity.
	\item Identify the critical path (\textbf{longest path through the network})
	\item Update the CPM diagram as the project progresses.
\end{enumerate}

The critical path is the longest-duration path through the network. The significance of the critical path is that the activities lie on it cannot be delayed without delaying the project.

\begin{flushleft}
\textbf{Hamilton Path}:\end{flushleft} To tour the 10 biggest cities in the UK, starting in London (number 1) and finishing in Bristol (number 10)is an example of finding a Hamilton Path.~\cite{path}

\begin{flushleft}
\textbf{Hamilton Cycle}:\end{flushleft} The ``Knight's Tour''~\cite{knight} is a sequence of moves done by a knight on a chessboard. The knight is placed on an empty chessboard and, following the rules of chess, must visit each square exactly once. The Knight's Tour problem is an instance of the more general Hamiltonian path problem in graph theory. This problem of getting a closed Knight's Tour is similarly an instance of the Hamiltonian cycle problem.

\bibliographystyle{plain}
\bibliography{references}
\end{document}
