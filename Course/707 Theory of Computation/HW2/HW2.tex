%% Based on a TeXnicCenter-Template by Gyorgy SZEIDL.
%%%%%%%%%%%%%%%%%%%%%%%%%%%%%%%%%%%%%%%%%%%%%%%%%%%%%%%%%%%%%

%------------------------------------------------------------
%
\documentclass{article}%
%Options -- Point size:  10pt (default), 11pt, 12pt
%        -- Paper size:  letterpaper (default), a4paper, a5paper, b5paper
%                        legalpaper, executivepaper
%        -- Orientation  (portrait is the default)
%                        landscape
%        -- Print size:  oneside (default), twoside
%        -- Quality      final(default), draft
%        -- Title page   notitlepage, titlepage(default)
%        -- Columns      onecolumn(default), twocolumn
%        -- Equation numbering (equation numbers on the right is the default)
%                        leqno
%        -- Displayed equations (centered is the default)
%                        fleqn (equations start at the same distance from the right side)
%        -- Open bibliography style (closed is the default)
%                        openbib
% For instance the command
%           \documentclass[a4paper,12pt,leqno]{article}
% ensures that the paper size is a4, the fonts are typeset at the size 12p
% and the equation numbers are on the left side
%
\usepackage{amsmath}%
\usepackage{amsfonts}%
\usepackage{amssymb}%
\usepackage{graphicx}
%-------------------------------------------
\newtheorem{theorem}{Theorem}
\newtheorem{acknowledgement}[theorem]{Acknowledgement}
\newtheorem{algorithm}[theorem]{Algorithm}
\newtheorem{axiom}[theorem]{Axiom}
\newtheorem{case}[theorem]{Case}
\newtheorem{claim}[theorem]{Claim}
\newtheorem{conclusion}[theorem]{Conclusion}
\newtheorem{condition}[theorem]{Condition}
\newtheorem{conjecture}[theorem]{Conjecture}
\newtheorem{corollary}[theorem]{Corollary}
\newtheorem{criterion}[theorem]{Criterion}
\newtheorem{definition}[theorem]{Definition}
\newtheorem{example}[theorem]{Example}
\newtheorem{exercise}[theorem]{Exercise}
\newtheorem{lemma}[theorem]{Lemma}
\newtheorem{notation}[theorem]{Notation}
\newtheorem{problem}[theorem]{Problem}
\newtheorem{proposition}[theorem]{Proposition}
\newtheorem{remark}[theorem]{Remark}
\newtheorem{solution}[theorem]{Solution}
\newtheorem{summary}[theorem]{Summary}
\newenvironment{proof}[1][Proof]{\textbf{#1.} }{\ \rule{0.5em}{0.5em}}

\begin{document}

\begin{flushleft}
\textbf{Course:} CSC707, Automata, Computability and Computational Theory\\
\textbf{Homework 2}: Complexity Theory, polynomial time reduction, P vs NP, NP-hard and NP-complete problems. \\
\textbf{Submission:} Use Wolfware\\
\textbf{File Format:} LaTeX and PDF\\
\end{flushleft}

\begin{center}
\fbox{\textbf{Due Date:} \textbf{2:00 A.M. (EST), Thursday, February 11, 2010}}\\
\begin{enumerate}
	\item Provide any feedback/questions you may have on this homework (\textbf{optional}).
	\item Using LaTeX is required.
\end{enumerate}\end{center}

\noindent{\hrulefill}

\bigskip

\begin{enumerate}

	\item Let $\sum$ denote $\{0,1\}$. Let $\propto $ denote polynomial-time reducibility. A set $L \subseteq \sum^*$ is $P$-complete if $L \in P$ and $M \propto L$ for all $M$, where $M \subseteq \sum^*$ and $M \in P$.
	\begin {enumerate}
	\item Show that $\emptyset$ and $\sum^*$ are not $P$-complete.
	\item Show that if $L \in P$ and $\emptyset \neq L \neq \sum^*$, then $L$ is  $P$-complete.  
	\end{enumerate}


  \item Show that the Vertex Cover remains $NP$-complete even when all the vertices in the graph 	are restricted to have even degree.\\
  Vertex Cover is defined as follows:\\
  INSTANCE: A graph $G=(V,E)$ and a positive integer $k \leq |V|$.\\
  QUESTION: Is there a subset $V' \subseteq V$ such that $|V'| \leq k$, and for each edge $\{u,v\} \in E$ at least one of $u$ and $v$ belongs to $V'$?
  
	\item Show that the Set Cover problem is $NP$-complete using the reduction from Vertex Cover.\\
  Set Cover problem is defined as follows:\\
  INSTANCE: A set $X$ of $n$ elements, a family $F$ of subsets of $X$, and a positive integer $k$.\\
  QUESTION: Is there a set $k$ or fewer subsets from $F$  whose union is $X$?\\
  For example, if $X=\{1,2,3,4\}$ and $F=\{\{1,2\}, \{2,3\}, \{4\}, \{2,4\}\}$, a solution does NOT exist for $k=2$ but does exist for $k=3$ (e.g., $\{\{1,2\}, \{2,3\}, \{4\}\}$.
  
  	\item The Independent Set problem is defined as follows.\\
  Set Cover problem is defined as follows:\\
  INSTANCE: A graph $G=(V,E)$ and a positive integer $k \leq |V|$.\\
  QUESTION: Does $G$ contain an indepenent set of size $k$ or more, i.e., a subset $V' \subseteq V$ and  $|V'| \geq k$ such that no two vertices in $V'$ are joined by an edge in $E$?\\
  Suppose you are given a graph, $G=(V,E)$, and an integer $k$ as input with $|V|=n$. And suppose you are given an algorithm, $D$, that solves the decision version of the Independent Set problem in time $T(n,k)$.
  	\begin {enumerate}
	\item Use $D$ to find the size of the maximum independent set, and state the time complexity involved.
	\item Use $D$ in a self-reduction to solve the search version of the independent set problem, and state the time complexity involved.  
	\end{enumerate}
	\end{enumerate}

\end{document}
