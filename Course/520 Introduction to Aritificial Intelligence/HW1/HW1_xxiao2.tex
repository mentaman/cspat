%% Based on a TeXnicCenter-Template by Gyorgy SZEIDL.
%%%%%%%%%%%%%%%%%%%%%%%%%%%%%%%%%%%%%%%%%%%%%%%%%%%%%%%%%%%%%

%------------------------------------------------------------
%
\documentclass{article}%
%Options -- Point size:  10pt (default), 11pt, 12pt
%        -- Paper size:  letterpaper (default), a4paper, a5paper, b5paper
%                        legalpaper, executivepaper
%        -- Orientation  (portrait is the default)
%                        landscape
%        -- Print size:  oneside (default), twoside
%        -- Quality      final(default), draft
%        -- Title page   notitlepage, titlepage(default)
%        -- Columns      onecolumn(default), twocolumn
%        -- Equation numbering (equation numbers on the right is the default)
%                        leqno
%        -- Displayed equations (centered is the default)
%                        fleqn (equations start at the same distance from the right side)
%        -- Open bibliography style (closed is the default)
%                        openbib
% For instance the command
%           \documentclass[a4paper,12pt,leqno]{article}
% ensures that the paper size is a4, the fonts are typeset at the size 12p
% and the equation numbers are on the left side
%
\usepackage{amsmath}%
\usepackage{amsfonts}%
\usepackage{amssymb}%
\usepackage{graphicx}
%-------------------------------------------
\newtheorem{theorem}{Theorem}
\newtheorem{acknowledgement}[theorem]{Acknowledgement}
\newtheorem{algorithm}[theorem]{Algorithm}
\newtheorem{axiom}[theorem]{Axiom}
\newtheorem{case}[theorem]{Case}
\newtheorem{claim}[theorem]{Claim}
\newtheorem{conclusion}[theorem]{Conclusion}
\newtheorem{condition}[theorem]{Condition}
\newtheorem{conjecture}[theorem]{Conjecture}
\newtheorem{corollary}[theorem]{Corollary}
\newtheorem{criterion}[theorem]{Criterion}
\newtheorem{definition}[theorem]{Definition}
\newtheorem{example}[theorem]{Example}
\newtheorem{exercise}[theorem]{Exercise}
\newtheorem{lemma}[theorem]{Lemma}
\newtheorem{notation}[theorem]{Notation}
\newtheorem{problem}[theorem]{Problem}
\newtheorem{proposition}[theorem]{Proposition}
\newtheorem{remark}[theorem]{Remark}
\newtheorem{solution}[theorem]{Solution}
\newtheorem{summary}[theorem]{Summary}
\newenvironment{proof}[1][Proof]{\textbf{#1.} }{\ \rule{0.5em}{0.5em}}

\begin{document}

\begin{flushleft}
\textbf{Course:} CSC520, Introduction to Artificial Intelligence\\
\textbf{Homework 1}\\
\textbf{Student: Xusheng Xiao} \\
\textbf{Unity ID: xxiao2} \\
\textbf{Email: xxiao2@ncsu.edu}
\end{flushleft}

\noindent{\hrulefill}

\bigskip

\begin{enumerate}
	\item \textbf{ (12 points) Describe PEAS for the following:}
	\begin{enumerate}
	\item Bot to display advertisements in a search engine (eg. Bing, Google etc.)
	\item Industrial robot (eg. detect surface defects on automobile body in assembly line)
	\item Recommendation system (eg. Amazon book suggestion system)
	\end{enumerate}
     
     In each case, state whether the environment is fully observable, deterministic, episodic, and single agent. \\
     

	\begin{tabular}{|p{2.5cm}|p{2.5cm}|p{2.5cm}|p{2cm}|p{2cm}|}
	\hline  \textbf{Agent Type} &  \textbf{Performance Measure }&  \textbf{Environment }& \textbf{Actuators }& \textbf{Sensors}  \\ 
	\hline  AD Bot in search engines &  Highly relevant to search condition, low prize, free shipping &  users, keyboards,  screen & Display ADs  & Keyboard entry of search condition and search history\\ 
	\hline  Industrial robot  & Maximize precision of detecting defects, low cost & Product parts, assemble line & Display detected defects & Scanned image data \\ 
	\hline  Recommendation system &  Recommended books are highly related to user preference and most possible for users to purchase  &  Users, recommendation system & Display recommended books  &Keyboard entry of search words, history of purchased book\\ 
	\hline 
	\end{tabular} 
	
	\begin{tabular}{|p{2.5cm}|p{2.5cm}|p{2.5cm}|p{2cm}|p{2cm}|}
	\hline  \textbf{Task Environment} &  \textbf{Observable}&  \textbf{Deterministic }& \textbf{Episodic }& \textbf{Agents}  \\ 
	\hline  AD Bot in search engines &  Partially &  Stochastic & Sequential  & single agent\\ 
	\hline  Industrial robot  & Fully & Deterministic & Episodic & single agent \\ 
	\hline  Recommendation system &  Partially  &  Stochastic & Sequential  &single agent\\ 
	\hline 
	\end{tabular} 



\item \textbf{(18 points) Answer the following questions from the textbook: 2.6a, 2.6b, 2.12}

	\begin{enumerate}
	\item This exercise explores the difference between agent functions and agent programs.

		\begin{enumerate}
		\item Can there be more than one agent program that implements a given agent function? Given an example, or show why one is not possible.
		\item Are there agent functions that cannot be implemented by any agent program?
		\end{enumerate}

\textbf{	Answer:}
		\begin{enumerate}
		\item There can be multiple programs that implement a given agent function. For example, for the vacuum agent problem shown in textbook, using table to map the percept to action is one implementation. However, we can also compute the hash code of the percept by adopting some hash function, and then use the computed hash code to look up for the corresponding action. This hash-look-up is another implementation for the same vacuum agent problem. Thus,  there can be multiple programs that implement a given agent function. 
		\item I believe the answer is "there is not any function cannot be implemented by any agent program". The scientific research is evolving fast in modern society. A few years ago, touch screen only can support very basic functionalities for human to communicate with computers. Using just several fingers to do zooming and rotating seems like an agent function that cannot be implemented. Nowadays, with the advances of multi-touch technologies, users now can touch the screen with multiple fingers and using different combination of fingers to send different signals to PC for different actions, such as zooming and rotating objects. 
		\end{enumerate}

	\item Consider a modified version of Exercise 2.8, in which the geography of the environment - its extent, boundaries, and obstacles - is unknown, as is the initial dirt configuration. (The agent can go \textit{Up} and \textit{Down} as well as \textit{Left} and \textit{Right}.) Repeat this exercise for the case in which the location sensor is replaced with a ``bump'' sensor that detects the agent's attempts to move into obstacles or to cross the boundaries of the environment. Suppose the bump sensor stops working; how should the agent behave? 
	\begin{enumerate}
	\item Can a simple reflex agent be perfectly rational for this environment? Explain.\\
	Since the bump sensor stops working, which makes the agent cannot get the correct information about obstacles, extent and boundaries, a simple reflex agent cannot work perfectly. For example, if we define a rule that if there is no dirt on the current location, move right. In the situation where the right location has an obstacle and moving right cannot achieve, the simple reflex agent cannot have the other reaction and only can stay in the same location forever. 
	
	\item Can a simple reflex agent with a \textit{randomized} agent function outperform a simple reflex agent? Design such an agent and measure its performance on several environments.\\
	With the bump sensor not working, a randomized agent should perform better than a simple reflex agent. The reason is that with the simple reflex agent, there must be a correct percept of the environment for it to do the right action. But with the bump sensor not working, a simple reflex agent cannot react correctly. However, for a randomized agent, it would pick random action based on the given percept. Thus, it is possible that it can correctly pick up the right action with the bump sensor not working. The design could be quite simple: The vacuum sucks up when it detects the current location has dirt; every time when the agent detects that there is no dirt on the current location, it randomly picks up a possible direction and move. 
	
	\item Can you design an environment in which your randomized agent will perform poorly? Show your results. \\
	To make a randomized agent perform poorly, we can design an environment that requires the agent to move several times of the same direction, say \textit{Up}. Since it is randomized, the probability for it to choose \textit{Up} action continuously for several times is very low. It is very possible that it moves up and then the other direction, which makes it difficult to move ``up'' for several times.
	
	\item Can a reflex agent with state outperform a simple reflex agent? Design such an agent and measure its performance on several environments. Can you design a rational agent of this type? \\
	A reflex agent with state cannot outperform a simple reflex agent. Since the bump sensor is not working, the agent cannot get the information of what the current location is and whether the neighbour locations have obstacles. As a result, even if the agent can save state, there is no enough percept the agent can obtain as a state. Therefore, a reflex agent with state cannot outperform a simple reflex agent.
	\end{enumerate}

\end{enumerate}


\item \textbf{(20 points) Introducing our agent Mr.Wuf who will help us moving things from one place to another. One day, Mr.Wuf is assigned a task of transferring a set of boxes one-by-one by lifting them from location A and placing them in location B inside a building. A signalling system says whether the agent is near its destination or not. The room has stationary obstacles whose locations are unknown. If the agent bumps into an obstacle, the box in hand will fall down and some boxes have fragile goods. But there are safe paths, some longer than the others. Your must help Mr.Wuf with the task by answering the following questions. Mr.Wuf does not have enough time !!!}

	\begin{enumerate}
	\item Define PEAS.
	\item Is it sufficient for Mr.Wuf to be simple reflex ? Why or why not ?
	\item Mr.Wuf likes to move randomly. To what extent would this help ? Are there drawbacks ?
	\item Suggest one improvement to Mr.Wuf's design. Does your improvement have drawbacks ?
	\end{enumerate}
	
	\textbf{	Answer:}
	\begin{enumerate}
	\item 
		\begin{tabular}{|p{2.5cm}|p{2.5cm}|p{2.5cm}|p{2.5cm}|p{2.5cm}|}
		\hline  Agent Type& Performance Measure &  Environment&  Actuators & Sensors \\ 
		\hline  Mr. Wuf & Find a shortest safe path for moving all the boxes from Location A to B in the building &  Location A \& B in a building, obstacles, boxes & hands for lifting boxes and legs for walking & Signalling system, eyes \\ 
		\hline 
		\end{tabular} 
	\item It is not sufficient. When the signalling system says that it is still not the destination, Mr. Wuf should move. Since Mr. Wuf is simple reflex, he can only choose one direction to move. If by moving in that direction he would face an obstacle, then he will still do that since he is simple reflex and can only move that direction by receiving the percept saying that current location is not destination. After he falls, he goes back to Location A and moves another box. However, since he is simple reflex, he would choose the same path as last time and fall again. As a result, he would fail the mission.
	\item Moving randomly can help Mr. Wuf a little. For example, when there is an obstacle on the right of Mr. Wuf, he may randomly choose to move up and then right to avoid that.
	\item Mr. Wuf can perform better if he is simple reflex with state and equipped with the sensor to detect obstacles and current locations. Whenever Mr. Wuf reaches a new location, save the location and the direction in a state. If he moves right on one location and then bump into an obstacle, he also save it. Later when he reaches the same location, he can change the direction clock-wisely and move. The drawback of it is it may not find an optimal path.
	\end{enumerate}
	
\item \textbf{(30 points) Consider the following english sentences.}
	\begin{enumerate}
	\item Marcus was a man
	\item Marcus was a Pompeian
	\item All Pompeians were Romans
	\item Caesar was a ruler
	\item All Romans were either loyal to Caesar or hated him
	\item Everyone is loyal to someone
	\item Any man only tries to assassinate rulers he is not loyal to
	\item Marcus tried to assassinate Caesar
	\end{enumerate}

\textbf{Now answer the following questions:}

	\begin{enumerate}
	\item Convert the above into first order predicate logic using the following predicates: ruler, man, Pompeian, Roman, loyalto, hate, tryassassinate Use appropriate quantifiers and connectors.
	\item Convert the above sentence to CNF.
	\item Answer the following questions using resolution discussed in class based on the knowledge above:
		\begin{enumerate}
		\item Was Marcus loyal to Caesar ?
		\item Did Marcus hate Caesar ?
		\end{enumerate}

	\end{enumerate}
	
	\textbf{Answer:}\\
	\begin{enumerate}
	\item FOPL:
		\begin{enumerate}
		\item $ man(Marcus) $
		\item $ Pompeian(Marcus) $
		\item $ \forall X (Pompeian(X) \Rightarrow Roman(X)) $
		\item $ ruler(Caesar) $
		\item $ \forall X (Roman(X) \Rightarrow loyalto(X, Caesar) \vee hate(X, Caesar)) $
		\item $ \forall X (\exists Y (loyalto(X, Y))) $
		\item $ \forall X (\forall Y ( (man(X) \wedge ruler(Y) \wedge tryassassinate(X,Y) ) \Rightarrow \neg loyalto(X,Y))) $
		\item $ tryassassinate(Marcus, Caesar) $
		\end{enumerate}
	\item CNF: 
		\begin{enumerate}
		\item $ man(Marcus) $
		\item $ Pompeian(Marcus) $
		\item $ \neg Pompeian(X1) \vee Roman(X1) $
		\item $ ruler(Caesar) $
		\item $ \neg Roman(X2) \vee loyalto(X2, Caesar) \vee hate(X2, Caesar) $
		\item $ loyalto(X3, f(X3)) -- f(X3)$ is a function of X3 
		\item $ \neg man(X4) \vee \neg ruler(Y1) \vee \neg tryassassinate(X4,Y1)  \vee \neg loyalto(X4,Y1) $
		\item $ tryassassinate(Marcus, Caesar) $
		\end{enumerate}
		
		\item \textbf{Answers:}
		\begin{enumerate}
		\item Conclusion: $  \neg loyalto(Marcus, Caesar)$\\
		------------------------------------------------\\
		Negated conclusion:  \\
		1. $ loyalto(Marcus, Caesar) $ \\
		------------------------------------------------\\
		\begin{tabular}{c|p{8cm}|l}
		2. &$ \neg ruler(Y1) \vee \neg tryassassinate(Marcus,Y1)  \vee \neg loyalto(Marcus,Y1) $ & $i + vii \lbrace Marcus/X4 \rbrace $\\ 
		3. & $ \neg tryassassinate(Marcus,Caesar)  \vee \neg loyalto(Marcus,Caesar)  $ &  $iv + 2 \lbrace Caesar / Y1 \rbrace$ \\
		4. &$ \neg loyalto(Marcus,Caesar) $& $viii + 3 \lbrace \rbrace$\\
		5. & $ \emptyset $ &$ 4 + 1 \lbrace \rbrace$\\

		\end{tabular} 
		
		\item Conclusion: $  hate(Marcus, Caesar)$\\
		------------------------------------------------\\
		Negated conclusion:  \\
		1. $ \neg hate(Marcus, Caesar) $ \\
		------------------------------------------------\\
		\begin{tabular}{c|p{8cm}|l}
		2. &$ \neg ruler(Y1) \vee \neg tryassassinate(Marcus,Y1)  \vee \neg loyalto(Marcus,Y1) $ & $i + vii \lbrace Marcus/X4 \rbrace $\\ 
		3. & $ \neg tryassassinate(Marcus,Caesar)  \vee \neg loyalto(Marcus,Caesar)  $ &  $iv + 2 \lbrace Caesar / Y1 \rbrace$ \\
		4. &$ \neg loyalto(Marcus,Caesar) $& $viii + 3 \lbrace \rbrace$\\
		5. & $ Roman(Marcus) $ &$ ii + iii \lbrace Marcus / X1 \rbrace$\\
		6. & $ loyalto(Marcus, Caesar) \vee hate(Marcus, Caesar)  $ &$ 5 + v \lbrace Marcus / X2 \rbrace$\\
		7. & $ hate(Marcus, Caesar)  $ &$ 6 + 4 \lbrace \rbrace$\\
		8. & $ \emptyset  $ &$ 7 + 1 \lbrace \rbrace$\\
		\end{tabular} 
		\end{enumerate} 
	\end{enumerate}
	


\item \textbf{(20 points) Consider the following English statements:}
	\begin{enumerate}
	\item John is a graduate student
	\item Graduate students buy cheaper books
	\item AI book is costly
	\item The neighborhood store "Bookmarks" has a discount on the AI book
	\item Books on discount are cheap
	\end{enumerate}

\textbf{Now, using the resolution approach for first order predicate logic discussed in class, answer: "Will John buy the AI book from BookMarks?"}

Lexicon: \\
Constant: John, AI book, Bookmarks \\
Variable: X,Y\\
Predicate: \\
\begin{tabular}{p{2.5cm}l}
 $graduate(X)$ & -- X is a graduate student \\ 
 $book(X)$ & -- X is a book  \\
 $cheap(X)$ & -- X is cheap  \\
$ buy(X,Y,Z)$ & -- X buys Y from Z\\
 $store(X)$ & -- X is a store  \\ 
 $discount(X)$ & -- X has discount  \\ 
 $ hasBook(X,Y) $ & -- X has book Y
\end{tabular} 

FOPL:
\begin{enumerate}
\item $ graduate(John) $
\item $ \forall X, Y, Z (graduate(X) \wedge book(Y) \wedge store(Z) \wedge hasBook(Z,Y)\wedge  cheap(Y)  \Rightarrow  buy(X,Y,Z)) $
\item $ book(AI book) \wedge \neg cheap(AI book) $
\item $ store(Bookmarks) \wedge hasBook(Bookmarks, AI book) \wedge discount(AI book) $
\item $ \forall X (book(X) \wedge discount(X) \Rightarrow cheap(X)) $
\end{enumerate}

CNF:
\begin{enumerate}
\item $ graduate(John) $
\item $ \neg graduate(X1) \vee \neg book(Y1) \vee \neg cheap(Y1) \vee \neg store(Z) \vee \neg hasBook(Z1,Y1) \vee buy(X1,Y1,Z1)$
\item $ book(AI book) $
\item $ \neg cheap(AI book) $
\item $ store(Bookmarks) $
\item $ hasBook(Bookmarks, AI book) $
\item $ discount(AI book) $
\item $ \neg book(X2) \vee \neg discount(X2) \vee cheap(X2) $
\end{enumerate}

Conclusion: $ buy(John, AI book, Bookmarks) $ -- John buy the AI book from BookMarks. \\
------------------------------------------------\\
Negated conclusion:  \\
1. $ \neg buy(John, AI book,Bookmarks) $ \\
------------------------------------------------\\
\begin{tabular}{c|p{8cm}|l}
2. & $ \neg discount(AI Book) \vee cheap(AI Book) $ & $h + c \lbrace AI Book/X2 \rbrace $\\ 
3. & $ cheap(AI Book) $ &  $g + 2 \lbrace  \rbrace$ \\
4. & $ \neg book(Y1) \vee \neg cheap(Y1)  \vee \neg store(Z1) \vee \neg hasBook(Z1,Y1) \vee buy(John,Y1,Z1)$& $b + a \lbrace John / X1 \rbrace$\\
5. & $ \neg cheap(AI Book) \vee \neg store(Z1) \vee \neg hasBook(Z1,AI book) \vee buy(John,AI Book, Z1) $ &$ 4 + c \lbrace AI Book / Y1 \rbrace$\\
6. & $ \neg store(Z1) \vee \neg hasBook(Z1,AI book) \vee buy(John,AI Book,Z1) $ &$ 5 + 3 \lbrace \rbrace$\\
7. & $ \neg hasBook(Bookmarks,AI book)  \vee buy(John, AI book,Bookmarks) $ &$ 6 + e \lbrace Bookmarks / Z1 \rbrace$\\ 
8. & $ buy(John, AI book, Bookmarks) $ &$ f + 7 \lbrace \rbrace$\\
9. & $ \emptyset $ &$ 1 + 8 \lbrace \rbrace$\\
\end{tabular} 

\end{enumerate}
\end{document}
