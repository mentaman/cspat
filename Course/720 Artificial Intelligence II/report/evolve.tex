\section{AI in Software Repair}
Manual fault fixing is a difficult, time-consuming, labor-intensive process. Some reports place software maintenance, traditionally defined as any modification made on a system after its delivery, at 90\% of the total cost of a typical software project~\cite{modernlegacy}. Modifying existing code, repairing defects,
and otherwise evolving software are major parts of those costs. 

To alleviate the burden, Weimer et al. ~\cite{repair,geneticPatch} introduce a fully automated method
for locating and repairing faults in software. Their approach works on off-the-shelf legacy applications and does not require formal specifications, program annotations or special coding practices. Once a program fault is discovered, an extended form of genetic programming~\cite{genericprogramming} is used to evolve program variants until one is found that both retains required functionality and also avoids the defect in question. Standard test cases are used to exercise the fault and to encode program requirements. After a successful repair has been discovered, it is minimized using structural differencing algorithms and delta debugging.

\subsection{Generic Programming}
Genetic programming (GP) is a computational method inspired by biological evolution, which discovers computer
programs tailored to a particular task~\cite{genericprogramming}. GP operates on and maintains a population comprised of different programs, referred to as individuals or chromosomes. In GP, each chromosome is a tree-based representation of a program. The fitness, or desirability, of each chromosome, is evaluated via an external fitness function. In their paper, fitness is assessed via the test cases. Once fitness is computed, high-fitness individuals are selected to be copied into the next generation. Variations are introduced through
computational analogies to the biological processes of mutation and crossover. These operations create
a new generation and the cycle repeats.

\subsection{Program Representation}
In their approach, each individual (candidate program) is represented as a pair containing:
(1) An abstract syntax tree (AST) including all of the statements in the program; (2) A weighted path through the program under test. The weighted path is a list of pairs, each pair containing a statement in the program and a weight based on that statement�s occurrences in various test cases.

\subsection{Insights}
A biggest impediment for an evolutionary algorithm like GP is the potentially infinite-size search space it must sample to search for a correct program. To address this problem, they introduce two key innovations. First, they restrict the algorithm to only produce changes that are based on structures in other parts of the program. In other words, they hypothesize that a program that is missing important functionality (e.g., a null check) will be able to copy and adapt it from another location in the program. Second, they constrain the
genetic operations of mutation and crossover to operate only on the region of the program that is relevant to the error (that is, the portions of the program that were on the execution path that produced the error). Theses two insights are used together to reduce the search space of software repair.

\subsection{Approach}
To automatically find patches for program repair, their approach uses GP to maintain a population of variants of that program. Each variant is represented as an abstract syntax tree (AST) paired with a weighted program path. Their approach modifies variants using two genetic algorithm operations, \textit{crossover} and \textit{mutation}, specifically targeted to this representation. Each modification produces a new abstract syntax
tree and weighted program path. Their approach then evaluates the fitness of each variant by compiling the abstract syntax tree and running it on the test cases. Its final fitness is a weighted sum of the positive and negative test cases it passes. Their approach stops when a program variant that passes all of the test cases is found. Because GP often introduces irrelevant changes or dead code, their approach further uses tree-structured difference algorithms and delta debugging techniques~\cite{deltadebugging} in a post-processing step to generate a 1-minimal patch that causes it to pass all of the testcases.

\subsection{Conclusion}
Weimer et al.~\cite{geneticPatch,repair} present a fully automated technique for repairing bugs in off-the-shelf legacy software. Rather than using formal specifications or special coding practices, their approach uses an extended form of genetic programming to evolve program variants. Their approach considers only certain classes of repairs, using one part of a program as a template to repair another part. In their GP algorithm, they use a novel representation that combines abstract syntax trees with weighted violating paths; these insights allow their search to scale to large programs. They use standard test cases to show the fault and to encode required functionality; their initial repair is a variant that passes all test cases. The initial repair is minimized using delta debugging and structural differencing algorithm. They are able to generate and minimize repairs for ten different C programs totalling 63,000 lines in under 200 seconds on average.










